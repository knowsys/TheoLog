% ⓒ 2017 Monika Sturm, Daniel Borchmann
% This work is licensed under the Creative Commons Attribution-ShareAlike 4.0
% International License. To view a copy of this license, visit
% http://creativecommons.org/licenses/by-sa/4.0/.

\documentclass[german]{latteachCD}[2017/03/28]

\usepackage[sheetnumber=4]{theolog}

\begin{document}

\maketitle

\begin{mdframed}
  Die folgenden Aufgaben werden nicht in den Übungen besprochen und dienen der
  Selbstkontrolle.

  \renewcommand{\theexercise}{\Alph{exercise}}
  \setcounter{exercise}{7}

  \begin{exercise}
    % Markus Krötzsch
    Sei $\Slang{L}$ eine unentscheidbare Sprache. Zeigen Sie:
    \begin{enumerate}
    \item\label{item:1} $\Slang{L}$ hat eine Teilmenge $\Slang{T}\subseteq \Slang{L}$, die
      entscheidbar ist.
    \item\label{item:2} $\Slang{L}$ hat eine Obermenge $\Slang{O}\supseteq
      \Slang{L}$, die entscheidbar ist.
    \item Es gibt jeweils nicht nur eine sondern unendlich viele entscheidbare
      Teilmengen bzw. Obermengen wie in~(\ref{item:1}) und~(\ref{item:2}).
    \end{enumerate}
  \end{exercise}

  \begin{exercise}
    % TheoLog 2014
    Gegeben ist eine Funktion $f \colon \mathbb N^{2} \to \mathbb N$ mit
    \begin{equation*}
      f(x,y) \coloneqq
      \begin{cases}
        1 & \text{falls $x \geq 1$ und $x$ teilt $y$,}\\
        0 & \text{falls $x \geq 1$ und $x$ teilt $y$ nicht,}\\
        \text{undefiniert} & sonst.
      \end{cases}
    \end{equation*}
    Geben Sie für $f$ ein WHILE-Programm an.
  \end{exercise}

\end{mdframed}

\vspace*{0.5\baselineskip}

\setcounter{exercise}{0}

\begin{exercise}
  % Aufgabe 5.11 von Sipser
  Zeigen Sie, dass eine Sprache $\Slang{L}$ genau dann entscheidbar ist, wenn
  $\Slang{L} \leq_{m} 0^{*}1^{*}$ gilt.
\end{exercise}

\begin{exercise}
  % Aufgabe 5.9 von Sipser
  Eine kontextfreie Grammatik $G$ heißt \emph{mehrdeutig}, wenn es ein Wort $w
  \in L(G)$ gibt, so dass es mehr als eine Ableitung von $w$ in $G$ gibt.
  Zeigen Sie, dass die Menge
  \begin{equation*}
    \Slang{P}_{\mathsf{ambigCFG}} \coloneqq \{\, \enc(G) \mid G \text{ ist eine
      mehrdeutige kontextfreie Grammatik}\,\}
  \end{equation*}
  unentscheidbar ist.  Nutzen Sie dazu eine Reduktion vom Postschen
  Korrespondenzproblem: ist $P$ eine Instanz vom PKP mit
  \def\BB#1{\begin{bmatrix}#1\end{bmatrix}}
  \begin{equation*}
    P = \BB{t_{1}\\b_{1}}, \dots, \BB{t_{k}\\b_{k}},
  \end{equation*}
  dann betrachte die kontextfreie Grammatik $G_{P}$ mit den Regeln
  \begin{align*}
    S &\to T \mid B\\
    T &\to t_{1}T\mathsf{a}_{1} \mid \dots \mid t_{k}T\mathsf{a}_{k} \mid
        t_{1}\mathsf{a}_{1} \mid \dots \mid t_{k}\mathsf{a}_{k}\\
    B &\to b_{1}B\mathsf{a}_{1} \mid \dots \mid b_{k}B\mathsf{a}_{k} \mid
        b_{1}\mathsf{a}_{1} \mid \dots \mid b_{k}\mathsf{a}_{k}
  \end{align*}
  wobei $\mathsf{a}_{1}, \dots, \mathsf{a}_{k}$ neue Terminalsymbole sind.
  Zeigen Sie dann, dass gilt
  \begin{equation*}
    P \text{ hat eine Lösung} \iff G_{P} \text{ ist mehrdeutig}.
  \end{equation*}
\end{exercise}

\begin{exercise}
  % evtl: Sipser 5.19
  \dots
\end{exercise}

\begin{exercise}
  % evtl: Sipser 5.35
  \dots
\end{exercise}

\end{document}
