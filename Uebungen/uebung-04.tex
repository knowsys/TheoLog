% ⓒ 2017 Monika Sturm, Daniel Borchmann
% This work is licensed under the Creative Commons Attribution-ShareAlike 4.0
% International License. To view a copy of this license, visit
% http://creativecommons.org/licenses/by-sa/4.0/.

\documentclass[german]{latteachCD}[2017/03/28]

\usepackage[sheetnumber=4]{theolog}

% Unentscheidbarkeit und PCP

\begin{document}

\maketitle

\begin{mdframed}
  Die folgende Aufgabe wird nicht in den Übungen besprochen und dienen der
  Selbstkontrolle.

  \renewcommand{\theexercise}{\Alph{exercise}}
  \setcounter{exercise}{6}

  \begin{exercise}
    % TheoLog 2014
    Welche der folgenden Aussagen sind wahr? Begründen Sie Ihre Antwort.
    \begin{enumerate}
    \item Die Menge der Instanzen des Postschen Korrespondenzproblems, welche
      eine Lösung haben, ist semi-entscheidbar.
    \item Das Postsche Korrespondenzproblem ist bereits über dem Alphabet
      $\Sigma = \{a, b\}$ nicht entscheidbar.
    \item Es ist entscheidbar, ob eine Turingmaschine nur Wörter akzeptiert, die
      Palindrome sind. (Ein Palindrom ist ein Wort $w = a_{1} \dots a_{n}$ mit
      $a_{1} \dots a_{n} = a_{n} \dots a_{1}$.)
    \item $\Slang{P}_{\mathsf{halt}}$ ist semi-entscheidbar.
    \item Es ist nicht entscheidbar, ob die von einer deterministischen
      Turing-Maschine berechnete Funktion total ist.
    \item Es gibt reguläre Sprachen, die nicht semi-entscheidbar sind.
    \end{enumerate}

    \vspace*{-0.5\baselineskip}
  \end{exercise}
\end{mdframed}

\vspace*{0.5\baselineskip}

\setcounter{exercise}{0}

\begin{exercise}
  % TheoLog 2014
  Besitzen folgende Instanzen $P_{i}$ des Postschen Korrespondenzproblems Lösungen oder
  nicht?  Begründen Sie Ihre Antwort.
  \begin{enumerate}
  \item $P_{1} = (a, aaa), (abaaa, ab), (ab, b)$
  \item $P_{2} = (ab, aba), (baa, aa), (aba, baa)$
  \item $P_{3} = (bba, b), (ba, baa), (ba, aba), (ab, bba)$
  \end{enumerate}
  (Für die dritte Teilaufgabe ist die Verwendung eines Computers sinnvoll.)
\end{exercise}

\begin{exercise}
  % TheoLog 2014
  Zeigen Sie, dass das Postsche Korrespondenzproblem über einem einelementigen
  Alphabet entscheidbar ist.
\end{exercise}

\begin{exercise}
  % Selbst
  Zeigen Sie, dass für eine gegebene semi-entscheidbare Sprache $L$ nicht
  berechenbar ist, was die minimale Anzahl an Zuständen einer Turing-Maschine
  ist, die $L$ erkennt.  Zeigen Sie dazu, dass für jedes $k$ die Menge
  \begin{equation*}
    T_{k} \coloneqq \{ \mathsf{enc}(\Smach{M}) \mid \mathcal{L}(\Smach{M})
    \text{ wird von einer TM mit höchstens $k$ Zuständen erkannt} \}
  \end{equation*}
  nicht entscheidbar ist.  Warum zeigt dies die ursprüngliche Behauptung?
\end{exercise}

\begin{exercise}
  % Sipser, Thm 5.30
  Zeigen Sie, dass weder das Äquivalenzproblem $\Slang{P}_{\mathsf{äquiv}}$ für
  Turing-Maschinen noch dessen Komplement
  $\overline{\Slang{P}}_{\mathsf{äquiv}}$ semi-entscheidbar ist, wobei
  \begin{align*}
    \Slang{P}_{\mathsf{äquiv}} \coloneqq \{ \enc(\Smach{M}_{1})\#\#\enc(\Smach{M}_{2})
    \mid \mathcal{L}(\Smach{M}_{1}) = \mathcal{L}(\Smach{M}_{2}) \}, \\
    \overline{\Slang{P}}_{\mathsf{äquiv}} \coloneqq \{ \enc(\Smach{M}_{1})\#\#\enc(\Smach{M}_{2})
    \mid \mathcal{L}(\Smach{M}_{1}) \neq \mathcal{L}(\Smach{M}_{2}) \}.
  \end{align*}
  Zeigen Sie dazu, dass $\Slang{P}_{\mathsf{halt}} \leq_{m}
  \Slang{P}_{\mathsf{äquiv}}$ und $\Slang{P}_{\mathsf{halt}} \leq_{m}
  \overline{\Slang{P}}_{\mathsf{äquiv}}$ gilt.  Weshalb zeigt dies die Aussage?
\end{exercise}

\end{document}

Aufgabe 3
