% ⓒ 2017 Monika Sturm, Francesco Kriegel, Daniel Borchmann
% This work is licensed under the Creative Commons Attribution-ShareAlike 4.0
% International License. To view a copy of this license, visit
% http://creativecommons.org/licenses/by-sa/4.0/.

\documentclass[german]{latteachCD}[2017/03/28]

\usepackage[sheetnumber=6]{theolog}
\title{Musterlösung zu Übungsblatt~10}

\usepackage{ragged2e}

\begin{document}

\maketitle

\vspace*{0.5\baselineskip}
\setcounter{exercise}{3}
\justifying

\begin{exercise}
  % TheoLog 2014
  Zeigen Sie, dass Allgemeingültigkeit von Formeln der Prädikatenlogik erster
  Stufe in Skolemform entscheidbar ist.
\end{exercise}

\emph{Lösung}\/: Es sei $F$ eine quantorenfreie Formel mit Variablen
$x_{1},\dots, x_{n}$.  Dann gilt
\begin{align*}
  \forall x_{1},\dots, x_{n}.\,F \text{ ist allgemeingültig}
  &\iff \exists x_{1},\dots,x_{n}.\,\neg F \text{ ist unerfüllbar}\\
  &\iff \neg F[x_{1}/a_{1},\dots,x_{n}/a_{n}] \text{ ist unerfüllbar}\\
  &\phantom{\iff}\:\: \text{(Skolemisierung mit Konstanten $a_{1},\dots,a_{n}$).}
\end{align*}

Es ist also $\forall x_{1},\ldots,x_{n}.\,F$ allgemiengültig genau dann, wenn
$\neg F[x_{1}/a_{1},\dots,x_{n}/a_{n}]$ unerfüllbar ist.  Letzteres ist aber
essentiell eine aussagenlogische Formel, und deren Erfüllbarkeit ist
entscheidbar.

\end{document}
