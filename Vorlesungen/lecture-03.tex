\newcommand*{\BeamerSuffix}{-overlay}
\newcommand*{\HandoutSuffix}{-print}
\PassOptionsToClass{onlymath}{beamer}
\documentclass[also={handout}]{beamerswitch}
\handoutlayout{nup=4}
\makeatletter
\beamer@ignorenonframefalse
\makeatother

\input{macros-lecture}

\defineTitle{3}{WHILE und LOOP}{12. April 2017}

\begin{document}

\maketitle

\begin{frame}\frametitle{Was bisher geschah \ldots}

\emph{Grundbegriffe, die wir verstehen und erklären können:}\\
DTM, NTM, Entscheider, Aufzähler, berechenbar/entscheidbar, semi-entscheidbar, unentscheidbar, Church-Turing-These
\bigskip

\emph{Das Unentscheidbare:}
\begin{itemize}
\item "`An algorithm is a finite answer to an infinite number of questions."' (Stephen Kleene)
\item Aber: Es gibt mehr Möglichkeiten, unendlich viele Fragen zu beantworten, als es Algorithmen geben kann (Cantor)
\end{itemize}

\bigskip
\emph{Weitere wichtige Ergebnisse:}
\begin{itemize}
\item DTM und NTM haben die gleiche Ausdrucksstärke
\item Zusammenhang Aufzähler $\leftrightarrow$ Semi-Entscheidbarkeit
\item Busy Beaver ist unentscheidbar: "`Was eine TM schaffen kann, das kann keine TM vorherberechnen."'
\end{itemize}







\end{frame}

\sectionSlide{LOOP}

\begin{frame}\frametitle{Von TMs zu Programmiersprachen}

\emph{Turingmaschinen als Berechnungsmodell}
\begin{itemize}
\item \alert{Pro:} Einfache, kurze Beschreibung (eine Folie)\\
	$\leadsto$ Beweise oft ebenfalls einfach und kurz
\item \redalert{Kontra:} Umständliche Programmierung\\
	$\leadsto$ einfache Algorithmen erfordern tausende Einzelschritte
\end{itemize}
\bigskip\pause

\emph{Programmiersprachen als Berechnungsmodell}
\begin{itemize}
\item \alert{Pro:} Einfache, bequeme Programmierung\\
	$\leadsto$ Großer Befehlssatz + Bibliotheken für Standardaufgaben
\item \redalert{Kontra:} Umständliche Beschreibung\\
	(z.B. Beschreibung von C++ [ISO/IEC 14882] hat 776 Seiten)\\
	$\leadsto$ Eigenschaften oft unklar; Beweise sehr umständlich
\end{itemize}

\end{frame}

\begin{frame}\frametitle{LOOP-Programme}

\emph{Idee:} Definiere eine imperative Programmiersprache, die dennoch sehr einfach ist.
\bigskip

\emph{Features:}
\begin{itemize}
\item Variablen \Sxsub{0}, \Sxsub{1}, \Sxsub{2}, \ldots oder auch \Svar{x}, \Svar{y}, \Svar{variablenName}, \ldots\\
alle vom Typ ``natürliche Zahl''
\item Wertezuweisungen der Form\\[1ex]
\mbox{}\hfill%
$\Svar{x} \Svassign \Svar{y} \Splus \Scodelit{42}\qquad$ und $\qquad\Svar{x} \Svassign \Svar{y} \Sminus \Scodelit{23}$\hfill\mbox{}\\[1ex]
%
für beliebige natürliche Zahlen und Variablennamen
\item ``For-Schleifen'': \SStartLoop{\Svar{x}} \ldots{} \SEndLoop
\end{itemize}

\end{frame}

\begin{frame}\frametitle{LOOP-Programme: Syntax}

\defbox{Die Programmiersprache \redalert{LOOP} basiert auf einer unendlichen Menge $\Slang{V}$ von Variablen
und der Menge $\mathbb{N}$ der natürlichen Zahlen.
\redalert{LOOP-Programme} sind induktiv definiert:
\begin{itemize}
\item Die Ausdrücke\vspace{-1ex}
\[\Svar{x} \Svassign \Svar{y} \Splus n \quad\text{und}\quad \Svar{x} \Svassign \Svar{y} \Sminus n \quad\text{\redalert{(Wertzuweisung)}}\]\vspace{-3.5ex}

sind LOOP-Programme für alle $\Svar{x},\Svar{y}\in\Slang{V}$ und $n\in\mathbb{N}$.
\item Wenn $P_1$ und $P_2$ LOOP-Programme sind, dann ist\vspace{-1ex}
\[P_1\Sseq{}~P_2 \quad\text{\redalert{(Hintereinanderausführung)}}\]\vspace{-3.5ex}

ein LOOP-Programm.
\item Wenn $P$ ein LOOP-Programm ist, dann ist\vspace{-1ex}
\[\SStartLoop{\Svar{x}}P~\SEndLoop{}\quad\text{\redalert{(Schleife)}}\]\vspace{-3.5ex}

ein LOOP-Programm, für jede Variable $\Svar{x}\in\Slang{V}$.
\end{itemize}

}
\end{frame}

\begin{frame}\frametitle{Beispiel}

Das folgende LOOP-Programm addiert zum Wert von \Svar{y} genau \Svar{x}-mal die Zahl $2$:
\bigskip

\codebox{{\tt
\SStartLoop{\Svar{x}}\\
~~~\Svar{y}\Svassign \Svar{y}\Splus\Scodelit{2}\\
\SEndLoop{}
}}\bigskip
\pause

Dies entspricht also der Zuweisung $\Svar{y}\Svassign \Svar{y}\Splus (\Scodelit{2}~\texttt{*}~\Svar{x})$, die wir in LOOP nicht direkt schreiben können.


\end{frame}

\begin{frame}\frametitle{LOOP-Programme: Semantik (1)}

\emph{Funktionsweise eines LOOP-Programms $P$:}
\begin{itemize}
\item \alert{Eingabe:} Eine Liste von $k$ natürlichen Zahlen\\
(Anmerkung: $k$ wird nicht durch das Programm festgelegt)
\item \alert{Ausgabe:} Eine natürliche Zahl
\end{itemize}
$P$ berechnet also eine totale Funktion $\mathbb{N}^k\to \mathbb{N}$, für beliebige $k$
\bigskip\pause

\emph{Initialisierung für Eingabe $n_1,\ldots,n_k$:}
\begin{itemize}
\item LOOP speichert für jede Variable eine natürliche Zahl als Wert
\item Den Variablen $\Sxsub{1},\ldots,\Sxsub{k}$ werden anfangs die Werte $n_1,\ldots,n_k$ zugewiesen
\item Allen anderen Variablen wird der Anfangswert $0$ zugewiesen

\end{itemize}

\end{frame}

\begin{frame}\frametitle{LOOP-Programme: Semantik (2)}

Nach der Initialisierung wird das LOOP-Programm abgearbeitet:
\begin{itemize}
\item $\Svar{x} \Svassign \Svar{y} \Splus n$:\\
der Variable $\Svar{x}$ wird als neuer Wert die Summe des (alten) Wertes für $\Svar{y}$ und der Zahl $n$ zugewiesen
\item $\Svar{x} \Svassign \Svar{y} \Sminus n$:\\
der Variable $\Svar{x}$ wird als neuer Wert die Differenz des (alten) Wertes für $\Svar{y}$ und der Zahl $n$ zugewiesen, falls diese größer $0$ ist; ansonsten wird $\Svar{x}$ der Wert $0$ zugewiesen
\item $P_1\Sseq{}~P_2$:\\
erst wird $P_1$ abgearbeitet, dann $P_2$
\item $\SStartLoop{\Svar{x}}P~\SEndLoop{}$:\\
$P$ genau $n$-mal ausgeführt, für die Zahl $n$, die $\Svar{x}$ anfangs als Wert zugewiesen ist\\
\textcolor{devilscss}{($n$ ändert sich also nicht, wenn $P$ den Wert von $\Svar{x}$ ändert)}
\end{itemize}

\end{frame}

\begin{frame}\frametitle{LOOP-Programme: Semantik (3)}

\emph{Ausgabe eines LOOP-Programms:}\\
\begin{itemize}
\item Das Ergebnis der Abarbeitung ist der Wert der Variable $\Sxsub{0}$ nach dem Beenden der Berechnung
\end{itemize}
\bigskip\pause

\theobox{Satz: LOOP-Programme terminieren immer nach endlich vielen Schritten.}

\pause
\emph{Beweis:} Die Behauptung gilt sicherlich für Wertzuweisungen.
\bigskip

Weitere Fälle:
\begin{itemize}
\item $P_1\Sseq{}~P_2$:\\
	wenn $P_1$ und $P_2$ nach endlich vielen Schritten terminieren, dann auch $P_1\Sseq{}~P_2$
\item $\SStartLoop{\Svar{x}}P~\SEndLoop{}$:\\
	für jede mögliche Zuweisung von $\Svar{x}$ wird $P$ endlich oft wiederholt;
	wenn $P$ in endlich vielen Schritten terminiert, dann also auch die Schleife\qed
\end{itemize}

\end{frame}

\begin{frame}\frametitle{Anmerkung: Strukturelle Induktion}

Der vorangegangene einfache Beweis verwendet \redalert{Induktion}, um eine
Aussage für unendlich viele Programme zu zeigen:
\begin{itemize}
\item \alert{Induktionsanfang:} Die Behauptung gilt für Wertzuweisungen (die einfachsten LOOP-Programme)
\item \alert{Induktionsannahme:} Die Behauptung gilt bereits für Programme $P$, $P_1$, $P_2$
\item \alert{Induktionsschritte:}
\begin{enumerate}[(1)]
\item Die Behauptung gilt dann auch für $P_1\Sseq{}~P_2$.
\item Die Behauptung gilt dann auch für $\SStartLoop{\Svar{x}}P~\SEndLoop{}$.
\end{enumerate}
\end{itemize}
\bigskip

\anybox{strongyellow}{Merke: Induktion kann man nicht nur auf natürliche Zahlen anwenden, sondern auf alle
(unendlichen) Mengen, die man induktiv mit endlich vielen Operationen aus Grundfällen erzeugen kann.
}

\end{frame}

\begin{frame}[t]\frametitle{Programmieren in LOOP (1)}

LOOP hat nur wenige Ausdrucksmittel, aber man kann sich leicht weitere als Makros definieren.\\[1ex]

\emph{Wertzuweisung mit Variable:} ``$\Svar{x}\Svassign \Svar{y}$'':\pause

\codebox{{\tt
\Svar{x}\Svassign \Svar{y}\Splus \Scodelit{0}
}}\bigskip\pause

\emph{Wertzuweisung mit $0$:} ``$\Svar{x}\Svassign \Scodelit{0}$'':\pause

\codebox{{\tt
\SStartLoop{\Svar{x}}\\
~~~\Svar{x}\Svassign \Svar{x}\Sminus\Scodelit{1}\\
\SEndLoop{}
}}\bigskip\pause

\emph{Wertzuweisung einer beliebigen konstanter Zahl:} ``$\Svar{x}\Svassign n$'':\pause

\codebox{{\tt
\Svar{x}\Svassign \Scodelit{0}\Sseq\\
\Svar{x}\Svassign \Svar{x}\Splus n
}}%\bigskip

\end{frame}

\begin{frame}[t]\frametitle{Programmieren in LOOP (2)}

LOOP hat nur wenige Ausdrucksmittel, aber man kann sich leicht weitere als Makros definieren.\\[1ex]

\emph{Wertzuweisung:} ``$\Svar{x}\Svassign \Svar{y}~\texttt{+}~\Svar{z}$'':\pause

\codebox{{\tt
\Svar{x}\Svassign\Svar{y}\Sseq\\
\SStartLoop{\Svar{z}}\\
~~~\Svar{x}\Svassign \Svar{x}\Splus\Scodelit{1}\\
\SEndLoop{}
}}\bigskip\pause


\emph{Fallunterscheidung:} ``$\SStartIf{\Svar{x}\Svneq\Scodelit{0}}~P~\SEndIf$'':\pause

\codebox{{\tt
\SStartLoop{\Svar{x}}~\Svar{y}\Svassign \Scodelit{1}~~\SEndLoop{}\Sseq\\
\SStartLoop{\Svar{y}}~P~~\SEndLoop{}
}}\medskip

Dabei ist $\Svar{y}$ eine frische Variable, die bisher nirgends sonst verwendet wird.

% \bigskip

\end{frame}

\begin{frame}\frametitle{LOOP-Berechenbare Funktionen}

\defbox{Eine Funktion $\mathbb{N}^k\to\mathbb{N}$ heißt \redalert{LOOP-berechenbar}, wenn es ein LOOP-Programm gibt, das die Funktion berechnet.}\bigskip
\pause

\examplebox{Beispiel: Die folgenden Funktionen sind LOOP-berechenbar:
\begin{itemize}
\item Addition: $\tuple{x,y}\mapsto x+y$ (gerade gezeigt)
\item Multiplikation: $\tuple{x,y}\mapsto x\cdot y$ (siehe Übung)
\item Potenz: $\tuple{x,y}\mapsto x^y$ (entsteht aus $\cdot$ wie $\cdot$ aus $+$)
\item und viele andere \ldots (max, min, div, mod, usw.)
\end{itemize}
}

\end{frame}

\begin{frame}\frametitle{LOOP jenseits von $\mathbb{N}$}

\pause
LOOP kann auch das $\Svar{x}$-te Bit der Binärkodierung von $\Svar{y}$ berechnen. Dadurch kann
man in LOOP (auf umständliche Weise) auch Daten verarbeiten, die keine Zahlen sind:
\begin{enumerate}[(1)]
\item Kodiere beliebigen Input binär
\item Evaluiere die Binärkodierung als natürliche Zahl und verwende diese als Eingabe
\item Dekodiere den Input im LOOP-Programm
\end{enumerate}
In diesem Sinne sind viele weitere Funktionen LOOP-berechenbar.
\bigskip\pause

\examplebox{Beispiele für LOOP-berechenbare Funktionen:
\begin{itemize}
\item das Wortproblem regulärer, kontextfreier und kontextsensitiver Sprachen
\item alle Probleme in \complclass{NP}, z.B. Erfüllbarkeit propositionaler Logik
\item praktisch alle gängigen Algorithmen (Sortieren, Suchen, Optimieren, \ldots)
\end{itemize}
}

\end{frame}

\begin{frame}\frametitle{Die Grenzen von LOOP}

\theobox{Satz: Es gibt berechenbare Funktionen, die nicht LOOP-berechenbar sind.}\pause

Das ist weniger überraschend, als es vielleicht klingt:\smallskip

\emph{Beweis:} Ein LOOP-Programm terminiert immer. Daher ist jede LOOP-berechenbare Funktion total. Es gibt aber auch nicht-totale Funktionen, die berechenbar sind (z.B. die ``partiellste'' Funktion, die nirgends definiert ist).\qed
% \bigskip\pause
% 
% \theobox{Satz: Es gibt berechenbare totale Funktionen, die nicht LOOP-berechenbar sind.}\pause
% 
% Das ist überraschend. Hilbert glaubte 1926 noch, dass alle Funktionen so berechnet werden können -- quasi ein erster Versuch der Definition von Berechenbarkeit.
% 
% {\tiny Hilbert definierte LOOP-Berechenbarkeit etwas anders, mithilfe sogenannter \redalert{primitiv rekursiver Funktionen}.
% }

\end{frame}


\begin{frame}\frametitle{LOOP-berechenbar $\neq$ berechenbar}

\theobox{Satz: Es gibt berechenbare totale Funktionen, die nicht LOOP-berechenbar sind.}\pause

Das ist überraschend. Hilbert glaubte 1926 noch, dass alle Funktionen so berechnet werden können -- quasi ein erster Versuch der Definition von Berechenbarkeit.

{\tiny Hilbert definierte LOOP-Berechenbarkeit etwas anders, mithilfe sogenannter \redalert{primitiv rekursiver Funktionen}.
}\pause\bigskip

Bewiesen wurde der Satz zuerst von zwei Studenten Hilberts:
\begin{itemize}
\item Gabriel Sudan (1927)
\item Wilhelm Ackermann (1928)
\end{itemize}\pause
Jeder der beiden gab eine Funktion an (Sudan-Funktion und Ackermann-Funktion), die nicht LOOP-berechenbar ist.
\bigskip

Unser Beweis verwendet eine etwas andere Idee \ldots

\end{frame}

\begin{frame}\frametitle{Fleißige Biber für LOOP}

Die \redalert{Länge} eines LOOP-Programms ist die Anzahl an
Zeichen, aus denen es besteht. Dazu nehmen wir an:
\begin{itemize}
\item Zahlen werden in ihrer Dezimalkodierung geschrieben
\item Variablen sind mit lateinischen Buchstaben und Ziffern benannt (wir sehen $\Sxsub{123}$ als Schreibweise für $\Svar{x123}$ an)
\end{itemize}\bigskip\pause

\defbox{Die Funktion $\bbfunc_{\textsf{LOOP}}:\mathbb{N}\to\mathbb{N}$ liefert für jede Zahl $\ell$ die größte Zahl $\bbfunc_{\textsf{LOOP}}(\ell)$, die ein LOOP-Programm der Länge $\leq\ell$ bei einer leeren Eingabe (alle Variablen sind $0$) ausgibt. Dabei sei
$\bbfunc_{\textsf{LOOP}}(\ell)=0$ falls es kein Programm der Länge $\leq\ell$ gibt.
}\pause

\emph{Beobachtung:} $\bbfunc_{\textsf{LOOP}}$ ist wohldefiniert:
\begin{itemize}
\item Die Zahl der LOOP-Programme mit maximaler Länge $\ell$ ist endlich
\item Unter diesen Programmen gibt es eine maximale Ausgabe
\end{itemize}

\end{frame}

\begin{frame}\frametitle{Beispiele}

\examplebox{Beispiel: Die LOOP-Anweisung
$\Sxsub{0}\texttt{:=}\Svar{y}\texttt{+}9$
liefert das fleißigste Programm für $\ell=7$, d.h. $\bbfunc_{\textsf{LOOP}}(7)=9$.
}\medskip\pause

Für $\ell=8$ gilt dementsprechend bereits $\bbfunc_{\textsf{LOOP}}(8)=99$.
\medskip

Für $\ell<7$ gibt es keine Zuweisung, die $\Sxsub{0}$ ändert, d.h., $\bbfunc_{\textsf{LOOP}}(\ell)=0$.
\bigskip\pause

\alert{Bonusaufgabe:} Gibt es eine Zahl $\ell$, bei der $\bbfunc_{\textsf{LOOP}}(\ell)$ durch ein
Programm berechnet wird, welches die Zahl $\bbfunc_{\textsf{LOOP}}(\ell)$ nicht
als Konstante im Quelltext enthält? Wie könnte das entsprechende Programm aussehen?


\end{frame}

\begin{frame}[t]\frametitle{Beweis (1)}

\theobox{Satz: Es gibt berechenbare totale Funktionen, die nicht LOOP-berechenbar sind.}\pause

\emph{Beweis:} Wir zeigen zwei Teilaussagen:

\begin{enumerate}[(1)]
\item $\bbfunc_{\textsf{LOOP}}$ ist berechenbar
\item $\bbfunc_{\textsf{LOOP}}$ ist nicht LOOP-berechenbar
\end{enumerate}\pause\medskip

Behauptung (1) ist leicht zu zeigen:
\begin{itemize}
\item Es gibt endlich viele LOOP-Programme der Länge $\leq \ell$
\item Man kann alle davon durchlaufen und auf einem Computer simulieren
\item Die Simulation liefert immer nach endlich vielen Schritten ein Ergebnis
\item Das Maximum aller Ergebnisse ist der Wert von $\bbfunc_{\textsf{LOOP}}(\ell)$
\end{itemize}
{\footnotesize (Anmerkung: Wir verwenden hier einen intuitiven Berechnungsbegriff und Church-Turing.)
}

\end{frame}

\begin{frame}[t]\frametitle{Beweis (2)}

\theobox{Satz: Es gibt berechenbare totale Funktionen, die nicht LOOP-berechenbar sind.}

\emph{Beweis:} Wir zeigen zwei Teilaussagen:

\begin{enumerate}[(1)]
\item $\bbfunc_{\textsf{LOOP}}$ ist berechenbar
\item $\bbfunc_{\textsf{LOOP}}$ ist nicht LOOP-berechenbar
\end{enumerate}\pause

Behauptung (2) zeigen wir per Widerspruch:\pause
\begin{itemize}
\item Angenommen $\bbfunc_{\textsf{LOOP}}$ ist LOOP-berechenbar durch Programm $P_{\bbfunc}$. Sei $k$ die Länge von $P_{\bbfunc}$.\pause
\item Wir wählen eine Zahl $m$ mit $m\geq k+17+\log_{10} m$ (immer \ghost{möglich)}\pause
\item Sei $P_m$ das Programm $\Sxsub{1}\texttt{:=}\Sxsub{1}\texttt{+}m$ (Länge: $7+\lceil\log_{10}m\rceil$)\pause
\item Sei $P_{++}$ das Programm $\Sxsub{0}\texttt{:=}\Sxsub{0}\texttt{+}\Scodelit{1}$ (Länge: $8$)\pause
\item Wir definieren $P = P_m\Sseq\, P_{\bbfunc}\Sseq\, P_{++}$.\\\pause
Die Länge von $P$ ist $k+17+\lceil\log_{10}m\rceil$ und damit $\leq m$.\\\pause
Aber $P$ gibt die Zahl $P_{\bbfunc}(m)+1$ aus. 
Widerspruch.\qed
\end{itemize}

\end{frame}

\sectionSlide{WHILE}

\begin{frame}\frametitle{Was fehlt?}

\alert{Frage:} Wieso ist LOOP zu schwach?\medskip\pause

\alert{Intuitive Antwort:} LOOP-Programme terminieren immer (zu vorhersehbar)
\bigskip

$\leadsto$ Wir brauchen ein weniger vorhersehbares Programmkonstrukt

\end{frame}

\begin{frame}\frametitle{WHILE-Programme: Syntax und Semantik}

\defbox{Die Programmiersprache \redalert{WHILE} basiert wie LOOP auf
Variablen $\Slang{V}$ und natürlichen Zahlen $\mathbb{N}$.\\
\redalert{WHILE-Programme} sind induktiv definiert:
\begin{itemize}
\item Jedes LOOP-Programm ist ein WHILE-Programm
\item Wenn $P$ ein WHILE-Programm ist, dann ist\vspace{-1ex}
\[\SStartWhile{\Svar{x}}P~\SEndWhile{}\]\vspace{-3.5ex}

ein WHILE-Programm, für jede Variable $\Svar{x}\in\Slang{V}$.
\end{itemize}
}\bigskip\pause

Semantik von $\SStartWhile{\Svar{x}}P~\SEndWhile{}$:\\
$P$ wird ausgeführt solange der aktuelle Wert von $\Svar{x}$ ungleich $0$ ist.\\
\textcolor{devilscss}{(dies hängt davon ab, wie $P$ den Wert von $\Svar{x}$ ändert)}
\medskip

Ansonsten werden WHILE-Programme wie LOOP-Programme ausgewertet.

\end{frame}

\begin{frame}\frametitle{WHILE: Beobachtungen}

Es ist möglich, dass ein WHILE-Programm nicht terminiert, z.B.\medskip

\codebox{{\tt
\Svar{x}\Svassign \Scodelit{1}\Sseq\\
\SStartWhile{\Svar{x}}\\
~~~\Svar{y}\Svassign \Svar{y}\Splus\Scodelit{2}\\
\SEndLoop{}
}}\bigskip
\pause

Wir können $\SStartLoop{\Svar{x}}P~\SEndLoop{}$
ersetzen durch:\medskip

\codebox{{\tt
\Svar{z}\Svassign \Svar{x}\Sseq\\
\SStartWhile{\Svar{z}}\\
~~~P\Sseq\\
~~~\Svar{z}\Svassign \Svar{z}\Sminus\Scodelit{1}\\
\SEndWhile{}
}}

(für ein frisches $\Svar{z}$)\bigskip

Also sind LOOP-Schleifen eigentlich nicht mehr nötig.

\end{frame}

\begin{frame}\frametitle{WHILE-Berechenbare Funktionen}

\defbox{Eine partielle Funktion $f:\mathbb{N}^k\to\mathbb{N}$ heißt \redalert{WHILE-berechenbar}, wenn es ein WHILE-Programm $P$ gibt, so dass gilt:
\begin{itemize}
\item Falls $f(n_1,\ldots,n_k)$ definiert ist, dann terminiert $P$ bei Eingabe $n_1,\ldots,n_k$ mit der Ausgabe $f(n_1,\ldots,n_k)$
\item Falls $f(n_1,\ldots,n_k)$ nicht definiert ist, dann terminiert $P$ bei Eingabe $n_1,\ldots,n_k$ nicht
\end{itemize}}\bigskip\pause

Das wichtigste Ergebnis zu WHILE ist nun das folgende:

\theobox{Satz: Eine partielle Funktion ist genau dann WHILE-berechenbar, wenn sie Turing-berechenbar ist.}

\end{frame}

\begin{frame}\frametitle{WHILE $\to$ TM}

\emph{Behauptung 1:} DTMs können WHILE-Programme simulieren:\pause
\medskip

% Wir skizzieren die Arbeitsweise der Turingmaschine:
\begin{itemize}
\item Wir verwenden eine Mehrband-TM, in der es für jede Variable im simulierten Programm ein eigenes Band gibt.\pause
\item Natürliche Zahlen werden auf den Bändern binär kodiert.\pause
\item DTMs können leicht (a) ein Band auf ein anderes kopieren, (b) die Zahl auf einem Band um eins erhöhen \\$\leadsto$ daraus kann man schon DTMs für $\Svar{x}\Svassign\Svar{y}\Splus n$ erzeugen\pause
\item Simulation von $\Svar{x}\Svassign\Svar{y}\Sminus n$ ist analog möglich (mit zusätzlichem Test auf Gleichheit mit $0$ beim Dekrementieren)\pause
\item Sequentielle Programmausführung $P_1\Sseq P_2$ wird direkt im Zustandsgraphen der DTM umgesetzt ("`Hintereinanderhängung"' von TMs)\pause
\item While-Schleifen sind durch Zyklen im Zustandsgraph darstellbar, wobei am Anfang jeweils ein Test auf Gleichheit mit $0$ steht, um die Schleife verlassen zu können
\end{itemize}


\end{frame}

\begin{frame}[t]\frametitle{TM $\to$ WHILE (1)}

\emph{Behauptung 2:} WHILE-Programme können DTMs simulieren:\pause
\medskip

\begin{itemize}
\item Wir nehmen zur Vereinfachung an, dass das \ghost{TM-Arbeitsalphabet}\\
$\Gamma=\{0,1\}$ ist, und dass die Zustände natürliche Zahlen \ghost{sind}
\item Eine TM-Konfiguration $a_1 a_2 \cdots a_p \,q\, a_{p+1}a_{p+2}\cdots a_{\ell}$ wird dargestellt durch drei Variablen:
\begin{itemize}
\item \Svar{left} hat den Wert, der durch $a_1 a_2 \cdots a_p$ binär kodiert wird (least significant bit ist dabei $a_p$)
\item \Svar{state} hat den Wert $q$
\item \Svar{thgir} hat den Wert, der durch $a_{\ell} \cdots a_{p+2} a_{p+1}$ binär kodiert wird (least significant bit ist also $a_{p+1}$)
\end{itemize}
\item Diese Kodierung kann leicht auf größere Arbeitsalphabete erweitert werden ($n$-äre statt binäre Kodierung)
% \item Das WHILE-Programm simuliert die TM-Übergänge auf diesen drei Variablen
\end{itemize}

\end{frame}


\begin{frame}[t]\frametitle{TM $\to$ WHILE (2)}

\emph{Behauptung 2:} WHILE-Programme können DTMs simulieren:
\medskip

\begin{itemize}
\item Wie gesagt:\\
\Svar{left} hat den Wert, der durch $a_1 a_2 \cdots a_p$ binär kodiert wird \pause
\item Wir wollen auf (die Binärkodierung von) \Svar{left} wie auf einen \alert{Stapel} (Keller, Stack) zugreifen:\pause
\begin{itemize}
\item \alert{Pop:} der folgende Pseudocode ist in WHILE (und LOOP) implementierbar 
\codebox{{\tt
\Svar{top}\Svassign \Svar{left}\texttt{ mod }\Scodelit{2}\Sseq\\
\Svar{left}\Svassign \Svar{left}\texttt{ div }\Scodelit{2}\\
}}\pause
\item \alert{Push:} der folgende Pseudocode ist in WHILE (und LOOP) implementierbar 
\codebox{{\tt
\Svar{left}\Svassign \Svar{left}\texttt{ * }\Scodelit{2}\Splus\Svar{top}
}}
\end{itemize}
\item Auf \Svar{thgir} kann man genauso zugreifen
\end{itemize}

\end{frame}


\begin{frame}[t]\frametitle{TM $\to$ WHILE (3)}

\emph{Behauptung 2:} WHILE-Programme können DTMs simulieren:
\medskip

\begin{itemize}
\item Wir haben das Band in zwei Stacks kodiert, mit den Zeichen links und rechts neben dem TM-Kopf an oberster Stelle\pause
\item Die TM-Simulation erfolgt jetzt in einer WHILE-Schleife
	$\SStartWhile{\Svar{halt}} \quad P_{\text{Einzelschritt}}~\quad \SEndWhile$
\item Das Programm $P_{\text{Einzelschritt}}$ führt einen Schritt aus:
\begin{itemize}
\item \Svar{thgir.pop()} liefert Zeichen an Leseposition
\item Durch eine Folge von IF-Bedingungen kann man für jede Kombination aus
Zustand $q$ (in \Svar{state}) und gelesenem Zeichen eine Behandlung festlegen
\item Schreiben von Symbol $a$ durch \Svar{thgir.push(\textit{a})}
\item Bewegung nach rechts: \Svar{left.push(thgir.pop())}
\item Bewegung nach links: \Svar{thgir.push(left.pop())}
\item Zustandsänderung durch einfache Zuweisung
\item Anhalten durch Zuweisung $\Svar{halt}\Svassign\Scodelit{0}$
\end{itemize}
\end{itemize}

\end{frame}

\begin{frame}[t]\frametitle{TM $\to$ WHILE (3)}

\emph{Behauptung 2:} WHILE-Programme können DTMs simulieren:
\medskip

\alert{Zusammenfassung:}
\begin{itemize}
\item Natürliche Zahlen simulieren Stacks der Bandsymbole links und rechts
\item Berechnungsschritte werden durch einfache Arithmetik implementiert (in LOOP möglich)
\item WHILE-Schleife arbeitet Schritte ab, bis die TM hält
\end{itemize}
\medskip

\alert{Was fehlt noch zum detaillierten Beweis?}
\begin{itemize}
\item Unsere Stack-Implementierung kann noch nicht mit dem leeren Stack umgehen $\leadsto$ zusätzliche Tests und Sonderfälle (bei einseitig unendlichem TM-Band asymmetrisch)
\item Für größere Arbeitsalphabete würde man statt Binärkodierung eine $n$-äre Kodierung verwenden\qed
\end{itemize}

\end{frame}

\begin{frame}\frametitle{Zusammenfassung und Ausblick}

WHILE-Programme können alle berechenbaren Probleme lösen\\
(ein weiteres Indiz für die Church-Turing-These)
\bigskip

LOOP-Programme können fast alle praktisch relevanten Probleme lösen,
aber nicht alle berechenbaren Probleme
\bigskip

Beweistechniken: strukturelle Induktion, Widerspruch durch Selbstbezüglichkeit (Busy Beaver), TM mit zwei Stacks simulieren\bigskip

Programme in LOOP und WHILE online testen:\\
\url{http://www.eugenkiss.com/projects/lgw/}\bigskip

\anybox{yellow}{
Was erwartet uns als nächstes?
\begin{itemize}
\item Relevantere Probleme
\item Reduktionen
\item Rice
\end{itemize}
}

\end{frame}

% \begin{frame}[t]\frametitle{Bildrechte}
% 
% Folie \ref{frame_hilbert}: gemeinfrei\\
% Folie \ref{frame_cantor}: Fotografie von 1870, gemeinfrei\\
% Folie \ref{frame_rado}: Ausschnitt aus einer Fotografie von 1928, \url{http://www.bibl.u-szeged.hu/sztegy/photo/778.jpg}, CC-By-SA 3.0\\
% 
% \end{frame}


\end{document}