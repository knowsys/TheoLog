\documentclass[aspectratio=1610,onlymath]{beamer}
% \documentclass[aspectratio=1610,onlymath,handout]{beamer}

\input{macros-lecture}

\defineTitle{21}{Datalog}{6. Juli 2018}

\begin{document}

\maketitle




\begin{frame}\frametitle{Zusammenfassung und Ausblick}

Beschränkt man Prädikatenlogik auf endliche Modelle, so gibt es kein vollständiges und korrektes Verfahren zum logischen Schließen -- dafür wird Erfüllbarkeit semi-entscheidbar\bigskip

Auswertungsproblem auf endlichen Modellen = Anfragebeantwortung in Datenbanken\\
(\PSpace-vollständig, aber sub-polynomiell bzgl. Datenbankgröße)\bigskip

Prädikatenlogik hat Grenzen: 
\begin{itemize}
\item bei der Modellierung (logisches Schließen), z.B. transitiver Abschluss
\item bei logischen Abfragen (Model Checking), z.B. Erreichbarkeit
\end{itemize}\bigskip

\anybox{yellow}{
Was erwartet uns als nächstes?
\begin{itemize}
\item Datalog und Logik höherer Ordnung
\item Gödel
\item Probeklausur und 2. Repetitorium
\end{itemize}
}

\end{frame}


% \begin{frame}[t]\frametitle{Bildrechte}
% 
% Folie \ref{frame_herbrand}: Fotografie von Natasha Artin Brunswick, 1931, CC-By 3.0
% 
% \end{frame}


\end{document}
