\documentclass[aspectratio=1610,onlymath]{beamer}
% \documentclass[aspectratio=1610,onlymath,handout]{beamer}

\input{macros-lecture}

\hyphenation{Exp-Time}

\defineTitle{22}{Datalog}{8. Juli 2024}

\begin{document}

\maketitle

\begin{frame}\frametitle{Rückblick: Formeln als Anfragen}

{\scriptsize

\begin{tabular}[t]{|l|l|}
\multicolumn{2}{@{}l}{linien:}\\
\hline
\textbf{Linie} & \textbf{Typ} \\
\hline
85 & Bus \\\hline
3 & Tram \\\hline
F1 & Fähre \\\hline
\ldots & \ldots\\\hline
\end{tabular}
\hspace{15mm}
\begin{tabular}[t]{|l|l|l|}
\multicolumn{2}{@{}l}{haltestellen:}\\
\hline
\textbf{SID} & \textbf{Name} & \textbf{Rollstuhl}\\
\hline
17 & Hauptbahnhof & true\\\hline
42 & Helmholtzstr. & true\\\hline
57 & Stadtgutstr. & true\\\hline
123 & Gustav-Freytag-Str. & false\\\hline
\ldots & \ldots & \ldots\\\hline
\end{tabular}

\begin{tabular}[t]{|l|l|l|}
\multicolumn{2}{@{}l}{verbindung:}\\
\hline
\textbf{Von} & \textbf{Zu} & \textbf{Linie}\\
\hline
57  & 42 & 85 \\\hline
17  & 789 & 3 \\\hline
\ldots & \ldots & \ldots\\\hline
\end{tabular}
\hspace{0.5cm}
% 
% \medskip
\ghost{
\begin{minipage}[t]{9cm}
\vspace{0.0cm}
Die einfache Arität der Prädikatenlogik wird durch ein\\ \redalert{Schema} mit Namen (und oft auch Datentypen) ersetzt:
\begin{itemize}
\item linien[Linie:string, Typ:string]
\item haltestellen[SID:int, Halt:string, Rollstuhl:bool]
\item verbindung[Von:int, Zu:int, Linie:string]
\end{itemize}
Relationale Algebra: Parameter (Spalten) durch Namen adressiert\\
Prädikatenlogik: Parameter durch Reihenfolge adressiert
\end{minipage}}}\vspace{1cm}

Die Anfrage \alert{$\exists z_{\text{Linie}}.(\textsf{verbindung}(x_\text{Von},x_\text{Zu}, z_{\text{Linie}})\wedge \textsf{linien}(z_{\text{Linie}},x_{\text{Typ}}))$}
entspricht einer (natürlichen) Join-Operation ($\wedge$) mit anschließender Projektion ($\exists$):
\alert{$\pi_{\text{Von},\text{Zu},\text{Typ}}(\textsf{verbindung}\bowtie\textsf{linien})$}.

\end{frame}


\begin{frame}\frametitle{Rekursive Anfragen}

\emph{Rückblick:} Mit Prädikatenlogik können nur lokale Eigenschaften getestet werden.\medskip

Nichtlokale Eigenschaften wie die Erreichbarkeit in Graphen sind praktisch relevant (speziell in Graphdatenbanken)\medskip

\redalert{Wie kann man solche Anfragen logisch ausdrücken?}\bigskip\pause

\emph{Idee:} Um beliebig weit zu schauen muss man Rekursion einführen.\bigskip

\examplebox{\emph{Beispiel:} Eine Haltestelle ist von Helmholtzstr. aus erreichbar, wenn
\begin{enumerate}[(1)]
\item sie die Haltestelle Helmholtzstr. ist, oder
\item sie neben einer Haltestelle liegt, die von Helmholtzstr. aus erreichbar ist.
\end{enumerate}}

\end{frame}


\begin{frame}\frametitle{Rekursion in Logik}

\examplebox{\emph{Beispiel:} Eine Haltestelle ist von Helmholtzstr. aus erreichbar, wenn
\begin{enumerate}[(1)]
\item sie die Haltestelle Helmholtzstr. ist, oder
\item sie neben einer Haltestelle liegt, die von Helmholtzstr. aus erreichbar ist.
\end{enumerate}}

\redalert{Wie kann man das in Logik ausdrücken?}\medskip\pause

\examplebox{\emph{Beispiel:} Das Prädikat $\text{Erreichbar}$ enthält alle von Helmholtzstr. erreichbaren Haltestellen, wenn die folgenden Formeln erfüllt sind:
\begin{enumerate}[(1)]
\item $\text{Erreichbar}(\Scodelit{\texttt{helmholtzstr}})$
\item $\forall x,y,z.\text{Erreichbar}(x)\wedge \text{verbindung}(x,y,z) \to \text{Erreichbar}(y)$
\end{enumerate}}

\end{frame}

\begin{frame}\frametitle{Model Checking?}

\examplebox{\emph{Beispiel:} Das Prädikat $\text{Erreichbar}$ enthält alle von Helmholtzstr. erreichbaren Haltestellen, wenn die folgenden Formeln erfüllt sind:
\begin{enumerate}[(1)]
\item $\text{Erreichbar}(\Scodelit{\texttt{helmholtzstr}})$
\item $\forall x,y,z.\text{Erreichbar}(x)\wedge \text{verbindung}(x,y,z) \to \text{Erreichbar}(y)$
\end{enumerate}}\medskip

Sei $\mathcal{Q}$ die Menge der Formeln (1) und (2). \pause
\begin{itemize}
\item Die Modelle von $\mathcal{Q}$ sind alle Interpretationen, in denen $\text{Erreichbar}$ \alert{mindestens} die von Helmholtzstr. aus erreichbaren Haltestellen enthält.
\item Eine gegebene Datenbankinstanz, betrachtet als Interpretation, ist normalerweise kein Modell von $\mathcal{Q}$, sofern es nicht schon eine entsprechende Tabelle für $\text{Erreichbar}$ gibt.
\end{itemize}
\redalert{$\leadsto$ Model Checking führt hier nicht zum gewünschten Ergebnis!}

\end{frame}

\begin{frame}\frametitle{Datenbanken als logische Theorien}

Wir nehmen daher eine andere Perspektive ein:
\begin{itemize}
\item Datenbankinstanzen $\Inter$ werden als \redalert{endliche Menge von Fakten} $\mathcal{F}_\Inter$ dargestellt:
\[ \mathcal{F}_\Inter= \{ p(c_1,\ldots,c_n)\mid c_1,\ldots,c_n\in\Slang{C}\text{ und }\tuple{c_1^\Inter,\ldots,c_n^\Inter}\in p^\Inter\}\]
\textcolor{devilscss}{(wie zuvor nimmt man vereinfachend oft an, dass $c^\Inter=c$ gilt)}
\item \redalert{Rekursive Anfragen} können wie im Beispiel als prädikatenlogische Formelmenge $\mathcal{Q}$ dargestellt werden
\item Ein Fakt $p(c_1,\ldots,c_n)$ ist eine \redalert{Antwort} auf die Anfrage $\mathcal{Q}$ auf der Datenbank $\Inter$ wenn gilt: $\mathcal{F}_\Inter,\mathcal{Q}\models p(c_1,\ldots,c_n)$.
\end{itemize}
Das heißt, wir fassen \alert{Anfragebeantwortung als prädikatenlogisches Schließen} auf (Entailment, nicht Model Checking).

\end{frame}

\begin{frame}\frametitle{Beispiel}

\examplebox{\emph{Beispiel:} Für die Anfrage
\begin{align*}
\mathcal{Q}=\{ & \text{Erreichbar}(\Scodelit{\texttt{helmholtzstr}}),\\
	& \forall x,y,z.\text{Erreichbar}(x)\wedge \text{verbindung}(x,y,z) \to \text{Erreichbar}(y)\}
\end{align*}
und die Datenbankinstanz $\Inter$ mit 
\begin{align*}
\mathcal{F}_\Inter = \{ & \text{verbindung}(\Scodelit{\texttt{helmholtzstr}},\Scodelit{\texttt{stadtgutstr}},\Scodelit{\texttt{85}})\\
	& \text{verbindung}(\Scodelit{\texttt{stadtgutstr}},\Scodelit{\texttt{räcknitzhöhe}},\Scodelit{\texttt{85}})\\
	& \text{verbindung}(\Scodelit{\texttt{räcknitzhöhe}},\Scodelit{\texttt{zellescher\_weg}},\Scodelit{\texttt{11}})\\
	& \text{verbindung}(\Scodelit{\texttt{schillerplatz}},\Scodelit{\texttt{körnerplatz}},\Scodelit{\texttt{61}})
\}
\end{align*}
folgen genau die Antworten $\text{Erreichbar}(\Scodelit{\texttt{helmholtzstr}})$, $\text{Erreichbar}(\Scodelit{\texttt{stadtgutstr}})$, $\text{Erreichbar}(\Scodelit{\texttt{räcknitzhöhe}})$ und $\text{Erreichbar}(\Scodelit{\texttt{zellescher\_weg}})$.
}

\end{frame}

\begin{frame}\frametitle{Sind rekursive Anfragen praktikabel?}

Wir wissen: prädikatenlogisches Schließen ist unentscheidbar.
\bigskip

\redalert{Ist die Beantwortung rekursiver Anfragen dann überhaupt möglich?}
\bigskip\pause

\alert{Ja, wenn wir uns auf bestimmte Formen von Anfragen beschränken.}
\bigskip

% $\leadsto$ mehr dazu beim nächsten Mal

\defbox{Eine \redalert{Datalog-Regel} ist eine Formel der folgenden Form
\[ \forall x_1,\ldots, x_\ell.\quad H\leftarrow B_1\wedge \ldots\wedge B_n \]
wobei $H$ und $B_i$ prädikatenlogische Atome sind und $x_1,\ldots, x_\ell$ eine Liste aller in den Atomen vorkommenden Variablen ist. $H$ heißt \redalert{Kopf} und $B_1\wedge \ldots\wedge B_n$ \redalert{Rumpf} der Regel.
Wir verlangen, dass jede Variable im Kopf auch im Rumpf vorkommt.
Ein \redalert{Fakt} ist eine variablenfreie Regel mit $n=0$.\medskip

Ein \redalert{Datalog-Programm} $P$ ist eine Menge von Datalog-Regeln (einschl.\ Fakten).}\medskip

Es ist üblich, die Allquantoren nicht explizit zu schreiben. Fakten schreibt man ohne $\leftarrow$.

\end{frame}

% \newcommand{\pred}[1]{\text{#1}}
\begin{frame}\frametitle{Beispiele}

\examplebox{Das vorige Beispiel war bereits ein Datalog-Programm:
\begin{align*}
\text{Erreichbar}(\Scodelit{\texttt{helmholtzstr}}) & \\
\text{Erreichbar}(y) & \leftarrow\text{Erreichbar}(x)\wedge \text{verbindung}(x,y,z)
\end{align*}
}\bigskip\pause

\examplebox{Ein umfangreicheres Beispiel:
\begin{align*}
\pred{vater}(\pred{alice},\pred{bob}) & \quad
\pred{mutter}(\pred{alice},\pred{carla})\quad
\pred{mutter}(\pred{evan},\pred{carla}) \quad
\pred{vater}(\pred{carla},\pred{david})\\
\pred{Elter}(x,y) &\leftarrow \pred{vater}(x,y)\qquad
\pred{Elter}(x,y) \leftarrow \pred{mutter}(x,y)\\
\pred{Vorfahre}(x,y) &\leftarrow \pred{Elter}(x,y)\\
\pred{Vorfahre}(x,z) &\leftarrow \pred{Elter}(x,y)\wedge \pred{Vorfahre}(y,z)\\
\pred{GleicheGeneration}(x,x) & \leftarrow \pred{Vorfahre}(x,y)\\
\pred{GleicheGeneration}(y,y) &\leftarrow \pred{Vorfahre}(x,y)\\
\pred{GleicheGeneration}(x,y) & \leftarrow \pred{Elter}(x,v)\wedge\pred{Elter}(y,w)\wedge\pred{GleicheGeneration}(v,w)
\end{align*}}

\end{frame}

\begin{frame}\frametitle{Auswertung von Datalog}

\alert{Wie kann man logische Schlüsse aus Datalog ziehen?}\medskip

\emph{Idee:} Wende Regeln iterativ auf gegebene Fakten an, um neue Fakten abzuleiten

\begin{itemize}
\item Wir betrachten hier Datenbanken (Interpretationen) als logische Theorie, d.h., Menge von Grundfakten
\item Die gegebenen Fakten kann man sich als Teil eines Datalog-Programms vorstellen
\end{itemize}\medskip\pause

\defbox{Der \redalert{Konsequenzoperator} $T_P$ für ein Datalog-Programm $P$ 
bildet endliche Mengen von Fakten $\Inter$ auf Mengen von Fakten ab:
\[T_P(\Inter) = \{ H\theta \mid H\leftarrow B_1\wedge\ldots\wedge B_n\in P \text{ und es gibt Substitution $\theta$ mit }B_1\theta,\ldots,B_n\theta\in\Inter \}\]
}\pause

\emph{Beobachtungen:}
\begin{itemize}
\item Substitutionen $\theta$ mit $B_1\theta,\ldots,B_n\theta\in\Inter$ sind einfach Antworten auf die Datenbankanfrage $B_1\wedge\ldots\wedge B_n$ über Datenbank $\Inter$ \alert{$\leadsto$ praktisch berechenbar}
\item $\theta$ muss alle Variablen in der angewendeten Regel auf Konstanten (Domänenelemente) aus $\Inter$ abbilden
\end{itemize}

\end{frame}

\begin{frame}\frametitle{$T_P$ iterativ anwenden}

Zur Ermittlung aller Schlüsse muss man $T_P$ iterativ anwenden:\medskip

\defbox{Für ein Datalog-Programm $P$ definieren wir rekursiv:
\begin{itemize}
\item $T_P^0 = \emptyset$
\item $T_P^{i+1} = T_P(T_P^i)$ für alle $i\geq 0$
\end{itemize}}\medskip\pause

\emph{Beobachtungen:} 
\begin{itemize}
\item $T_P^1=T_P(\emptyset)$ ist die Menge aller Fakten in $P$
\item $T_P^i\subseteq T_P^{i+1}$ (frühere Regelanwendungen bleiben weiterhin möglich; Monotonie)
\item $T_P^i$ enthält nur Fakten, die man bilden kann, indem man einen Regelkopf aus $P$ mit Konstanten aus $P$ instanziiert $\leadsto$ endlich viele mögliche Schlüsse
\end{itemize}
\medskip\pause

\defbox{$T_P$ erreicht also nach endlich vielen Schritten einen Grenzwert, definiert wie folgt:
\[ T_P^\infty = \bigcup_{i\geq 0}T_P^i\]}

\end{frame}

\newcommand{\sgpred}{\pred{GG}}
\begin{frame}\frametitle{Beispiel}

\examplebox{Für das Programm $P$ mit den Regeln
\begin{align*}
\pred{vater}(\pred{alice},\pred{bob}) & \quad
\pred{mutter}(\pred{alice},\pred{carla})\quad
\pred{mutter}(\pred{evan},\pred{carla}) \quad
\pred{vater}(\pred{carla},\pred{david})\\
\pred{Elter}(x,y) &\leftarrow \pred{vater}(x,y)\qquad
\pred{Elter}(x,y) \leftarrow \pred{mutter}(x,y)\\
% \pred{Vorfahre}(x,y) &\leftarrow \pred{Elter}(x,y)\\
% \pred{Vorfahre}(x,z) &\leftarrow \pred{Elter}(x,y)\wedge \pred{Vorfahre}(y,z)\\
\sgpred(x,x) & \leftarrow \pred{Elter}(x,y)\\
\sgpred(y,y) &\leftarrow \pred{Elter}(x,y)\\
\sgpred(x,y) & \leftarrow \pred{Elter}(x,v)\wedge\pred{Elter}(y,w)\wedge\sgpred(v,w)
\end{align*}
\vspace{-4ex}

($\sgpred=\text{GleicheGeneration}$) ergibt sich:\medskip

{\footnotesize
% \begin{itemize}
% \item
$T_P^0=\emptyset$\\[0.9ex]
% \item
$T_P^1=\{ \pred{vater}(\pred{alice},\pred{bob}), \pred{mutter}(\pred{alice},\pred{carla}),\pred{mutter}(\pred{evan},\pred{carla}),\pred{vater}(\pred{carla},\pred{david}) \}$\\[0.9ex]
% \item 
$T_P^2=T_P^1\cup\{\pred{Elter}(\pred{alice},\pred{bob}), \pred{Elter}(\pred{alice},\pred{carla}),\pred{Elter}(\pred{evan},\pred{carla}),\pred{Elter}(\pred{carla},\pred{david}) \}$\\[0.9ex]
% \item 
$T_P^3=T_P^2\cup\{\sgpred(\pred{alice},\pred{alice}),\sgpred(\pred{bob},\pred{bob}),\sgpred(\pred{carla},\pred{carla}),\sgpred(\pred{david},\pred{david}),\sgpred(\pred{evan},\pred{evan})\}$\\[0.9ex]
% \item 
$T_P^4=T_P^3\cup\{\sgpred(\pred{alice},\pred{evan}),\sgpred(\pred{evan},\pred{alice})\}$\\[0.7ex]
% \item 
$T_P^5=T_P^4=T_P^\infty$
% \end{itemize}
}
}

\end{frame}

\begin{frame}\frametitle{Semantische Bedeutung von $T_P$}

\emph{Beobachtung 1:} Jede Datalog-Regel $H\leftarrow B_1\wedge\ldots\wedge B_n$ entspricht einer
Klausel $H\vee\neg B_1\vee\ldots\vee\neg B_n$, wobei jeweils alle Variablen allquantifiziert sind.
\medskip\pause

$\leadsto$ Datalog-Programme sind syntaktische Varianten skolemisierter Klauseln\\
$\leadsto$ Für die Inferenz von Fakten kann man sich auf Herbrand-Modelle beschränken\bigskip\pause

\emph{Beobachtung 2:} Die Berechnung eines Fakts $H\theta$ durch Anwendung einer solchen Regel entspricht einer (Hyper)-Resolution der Klausel $H\vee\neg B_1\vee\ldots\vee\neg B_n$ mit den Fakten $B_1\theta$, \ldots, $B_n\theta$, wobei $\theta$ der allgemeinste Unifikator ist.\medskip\pause

$\leadsto$ Abgeleitete Fakten sind logische Konsequenzen (Korrektheit)\bigskip\pause

Mithilfe dieser Einsichten lässt sich zeigen, dass $T_P^\infty$ zur Berechnung der logischen Schlussfolgerung geeignet ist:

\theobox{\emph{Satz:} Für ein Datalog-Programm $P$ ist $T_P^\infty$ das kleinste Herbrand-Modell. Das heißt, für einen beliebigen Fakt $F$ gilt $F\in T_P^\infty$ gdw. $F$ in jedem Herbrand-Modell vorkommt gdw. $P\models F$.}

\end{frame}

\begin{frame}\frametitle{Ableitungsbäume}

Die Folgerung $P\models F$ lässt sich als endlicher Baum darstellen:
\begin{itemize}
\item Jeder Knoten ist ein variablenfreies Atom
\item Jeder Elternknoten entsteht durch Anwendung einer Regel aus $P$ auf seine Kindknoten
\item Jeder Blattknoten ist ein gegebener Fakt aus $P$
\item Jeder Knoten wird zusätzlich mit der Regel und Substitution $\theta$ beschriftet, die zur Ableitung angewendet wurden
\end{itemize}
$\leadsto$ Ableitungsbaum stellt die Resolutionsableitung des Fakts an der Wurzel des Baums grafisch dar
\bigskip

\theobox{\emph{Beobachtung:} Für jeden Fakt $F\in T_P^\infty$ gibt es mindestens einen Ableitungsbaum mit Wurzel $F$.}

\end{frame}

\newcommand{\drule}[1]{\textcolor{darkred}{\scalebox{0.7}{$#1$}} }

\begin{frame}[t]\frametitle{Beispiel}

% We can think of deductions as tree structures:\\[3ex]

\def\svgwidth{12cm}
~\hspace{1.5cm}\resizebox{12cm}{!}{\import{images/}{proof-tree.pdf_tex}}
\vspace{-1.9cm}

{\framebox{\scalebox{0.7}{%\footnotesize
\begin{minipage}{6cm}%
\vspace{-5mm}
\begin{align*}%
(1) && \pred{vater}(\pred{alice},\pred{bob})\\
(2) && \pred{mutter}(\pred{alice},\pred{carla})\\
(3) && \pred{mutter}(\pred{evan},\pred{carla})\\
(4) && \pred{vater}(\pred{carla},\pred{david})\\
(5) && \pred{Elter}(x,y) &\leftarrow \pred{vater}(x,y)\\
(6) && \pred{Elter}(x,y) &\leftarrow \pred{mutter}(x,y)\\
(7) && \pred{Vorfahre}(x,y) &\leftarrow \pred{Elter}(x,y)\\
(8) && \pred{Vorfahre}(x,z) &\leftarrow \pred{Elter}(x,y)\wedge \pred{Vorfahre}(y,z)
\end{align*}
\end{minipage}}}}

\end{frame}

\begin{frame}\frametitle{Rückblick: Kontextfreie Grammatiken}

In der Vorlesung Formale Systeme haben wir auch eine Art von Ableitungsbäumen kennengelernt:
\alert{Ableitungsbäume für kontextfreie Grammatiken}\bigskip\pause

Diese sind eng mit denen aus Datalog verwandt, weil man kontextfreie Grammatiken in Datalog darstellen kann:\medskip

\emph{Gegeben: Kontextfreie Grammatik} $G=\tuple{V,\Sigma,P,\Snterm{S}}$
\begin{itemize}
\item Für jedes Nichtterminal $\Snterm{A}\in V$, sei $A$ ein zweistelliges Prädikat
\item Für jedes Terminal $\Sterm{b}\in \Sigma$, sei $b$ ein zweistelliges Prädikat
\item Für jede Produktionsregel $\Snterm{A}\to v$ mit $v=v_1\ldots v_n\in(\Sigma\cup V)^*$ sei $A(x_0,x_n)\leftarrow v_1(x_0,x_1)\wedge\ldots\wedge v_n(x_{n-1},x_n)$ eine Datalogregel
\item Die Menge aller entsprechenden Regel bezeichnen wir mit $P_G$
\end{itemize}
Wir betrachten Datenbanken, die nur Prädikate aus $\Sigma$ enthalten\bigskip\pause

Dann gilt: $P_G\models S(a,b)$ genau dann wenn es zwischen $a$ und $b$ einen
Pfad $w_1(a,c_1),w_2(c_1,c_2),\ldots,w_\ell(c_{\ell-1},b)$ gibt, wobei $\Sterm{w_1}\cdots \Sterm{w_\ell}\in\Slang{L}(G)$

% briefly discuss use of Datalog derivation trees in pumping-lemma like constructions ...

\end{frame}

\begin{frame}\frametitle{Wozu Ableitungsbäume?}

Baumartige Darstellungen von logischen Ableitungen sind in verschiedenen Zusammenhängen hilfreich:
\begin{itemize}
\item Intuitive Darstellung der Ableitungsschritte
\item Umformungsoperationen auf Bäumen erlauben die Erzeugung neuer Ableitungen $\leadsto$ hilfreiche Beweistechnik
\examplebox{\emph{Beispiel:} Wir haben Ableitungsbäume kontextfreier Grammatiken verwendet um das Pumpinglemma für kontextfrei Sprachen zu zeigen. Damit konnte man zeigen, dass manche Sprachen nicht kontextfrei sind.\medskip

Analoge Techniken kann man auch für Datalog anwenden, zum Beispiel um zu zeigen, dass kein Datalog-Programm Paare von Elementen finden kann, zwischen denen es einen Pfad aus einem binären Prädikat $p$ gibt, dessen Länge eine Primzahl ist.}
\item Zur Analyse von Datalogprogrammen werden oft (allgemeinere) endliche Automaten verwendet, welche auf Ableitungsbäumen arbeiten (Verallgemeinerung regulärer Sprachen von Wörtern auf Bäumen).
\end{itemize}

\end{frame}

\begin{frame}\frametitle{Komplexität}

Mithilfe von $T_P$ kann man logische Konsequenzen berechnen\\
\redalert{Wie aufwändig ist das?}\bigskip\pause

\emph{Worst Case?}
\begin{itemize}
\item Sei $p$ die Anzahl der Prädikatensmbole, $a$ deren maximale Arität, $x$ die maximale Zahl an Variablen pro Regel und $n$ die Zahl der Konstanten
\item Dann gibt es insgesamt $\leq p\cdot n^a$ variablenfreie Fakten, die abgeleitet werden könnten
\item Für eine Regel gibt es maximal $n^x$ Substitutionen, die bei der Ableitung eine Rolle spielen könnten
\end{itemize}\pause
$\leadsto$ Berechnung von $T_P^\infty$ in exponentieller Zeit möglich
\bigskip\pause

Man kann zeigen, dass dies worst-case-optimal ist:\medskip

\theobox{\emph{Satz:} Das Problem der Schlussfolgerung von Fakten (``$P\models F$?'') für Datalog
ist \ExpTime-vollständig.}

\end{frame}

\begin{frame}\frametitle{Ist Datalog praktisch?}

\redalert{\ExpTime ist eine ziemlich hohe Komplexität -- Ist Datalog praktisch implementierbar?}\pause\bigskip

\alert{Ja!}
\begin{itemize}
\item Die Worst-Case-Komplexität erfordert, dass die Stelligkeit von Prädikaten unbeschränkt wachsen kann $\leadsto$ in Anwendungen sind sehr große Stelligkeiten untypisch
\item In Abhängigkeit von der Größe der Datenbank (der Zahl der Fakten) wächst die Laufzeit lediglich polynomiell $\leadsto$ gutes Skalierungsverhalten für große Datenmengen
\item Es gibt inzwischen eine Reihe sehr effizienter Implementierungen, z.B.
\begin{itemize}
\item hochskalierbare speicherbasierte Systeme: z.B. Nemo (TU Dresden; \alert{\href{https://tools.iccl.inf.tu-dresden.de/nemo/}{online demo}}), RDFox (Oxford), Souffl\'{e}
\item Systeme basierend auf relationalen Datenbanken: z.B. Llunatic
\item Systeme für komplexere Logiken, die Datalog als Sonderfall unterstützen: z.B. clingo, DLV (Answer Set Programming), E (Theorembeweiser)
\end{itemize}
\end{itemize}

\end{frame}


\begin{frame}\frametitle{Logik höherer Ordnung}

Prädikatenlogik ist genau genommen \alert{Prädikatenlogik erster Stufe}.
\bigskip

\emph{Hintergrund:}
\begin{itemize}
\item Erste Stufe: Quantoren beziehen sich auf Domänenelemente\smallskip

\examplebox{\emph{Beispiel:} "`Jede natürliche Zahl $n$ hat einen Nachfolger $s(n)$"'}
% \[ \forall x.(\textsf{natnum}(x)\to\textsf{natnum}(s(x))) \]
\item Zweite Stufe: Quantoren beziehen sich auf Prädikate\smallskip

\examplebox{\emph{Beispiel:} "`Für jede Menge $M$ gilt: Enthält $M$ die Zahl $0$ und mit jeder natürlichen Zahl $n$ auch stets deren Nachfolger $s(n)$, so enthält $M$ alle natürlichen Zahlen."'}
\end{itemize}\bigskip

\emph{Logik zweiter Ordnung:}
\begin{itemize}
\item Ausdrucksstärker; kann z.B. die natürlichen Zahlen exakt charakterisieren
\item Schwieriger: kein vollständiges und korrektes \ghost{Beweisverfahren}
\end{itemize}

$\leadsto$ siehe Vorlesung \redalert{Advanced Logic}

\end{frame}

\begin{frame}\frametitle{Logik höherer Ordnung: Syntax und Semantik}

\emph{Syntax:} Wie in Prädikatenlogik, aber mit quantifizierten Prädikaten-Variablen.\\
\textcolor{devilscss}{(Die Stelligkeit einer Prädikaten-Variablen muss jeweils klar festgelegt werden)}\medskip

\examplebox{\emph{Beispiel:} ``Für jede Menge $M$ gilt: Enthält $M$ die Zahl $0$ und mit jeder natürlichen Zahl $n$ auch stets deren Nachfolger $s(n)$, so enthält $M$ alle natürlichen Zahlen.''
%
\[ \forall M.\big( M(0)\wedge \forall x.(M(x)\to M(s(x))\big) \to \forall y.M(y) )\]
%
Wir verwenden hier ein Funktionssymbol $s$ zur Darstellung von Nachfolgern.
}\bigskip\pause

\emph{Semantik:} wie erwartet (gleiche Interpretationen wie in erster Stufe; Interpretation von Prädikaten-Variablen mit Zuweisungen wie bei Objektvariablen in erster Stufe)\medskip

\textcolor{devilscss}{(Intuition: zweite Stufe/erste Stufe = QBF/Aussagenlogik)}

\end{frame}

\begin{frame}\frametitle{Logik höherer Ordnung: logisches Schließen}

Offenbar ist Schließen in Logik zweiter Stufe mindestens genauso schwer wie in Logik erster Stufe. Tatsächlich ist es noch deutlich schwerer:\medskip

\theobox{\emph{Fakt:} Logisches Schließen in Logik höherer Ordnung ist nicht semi-entscheidbar und insbesondere gibt es kein korrektes und vollständiges Ableitungsverfahren für diese Logik.}

\end{frame}

\begin{frame}\frametitle{Logik höherer Ordnung und Datalog}

\emph{Idee:} Die Fakten, die in allen Modellen (eines Datalog-Programms) gefolgert werden können, sind genau diejenigen, die in jeder erfüllenden Interpretation (in Logik zweiter Ordnung) der Datalog-Prädikate gelten.
\medskip

\examplebox{\emph{Beispiel:} Das Datalog-Programm
\begin{align*}
\text{Erreichbar}(\Scodelit{\texttt{helmholtzstr}}) & \\
\text{Erreichbar}(y) & \leftarrow\text{Erreichbar}(x)\wedge \text{verbindung}(x,y,z)
\end{align*}
kann als Formel der Logik zweiter Stufe wie folgt dargestellt werden:
\begin{align*}
\forall \text{Erreichbar}.\Big(&
(\text{Erreichbar}(\Scodelit{\texttt{helmholtzstr}}) \wedge{}\\
 &\hspace{-5mm}(\forall x,y,z.\text{Erreichbar}(x)\wedge \text{verbindung}(x,y,z)\to \text{Erreichbar}(y)) )\to \text{Erreichbar}(v)
\Big)
\end{align*}
Die Formel hat eine freie Variable $v$ und stellt $\text{Erreichbar}$ als Prädikaten-Variable dar.
Ein Fakt $\text{Erreichbar}(a)$ folgt aus dem ursprünglichen Programm über einer Datenbank $\Inter$ genau dann wenn $\Inter$ die Formel mit Variablenbelegung $v\mapsto a$ erfüllt.
}

\end{frame}

\begin{frame}\frametitle{Model Checking!}

Das vorige Beispiel zeigt:\medskip

\theobox{Die Beantwortung von Anfragen in Datalog entspricht einem Auswertungsproblem (Model Checking) für spezielle Formeln zweiter Ordnung über endlichen Modellen.}
\bigskip

Diese Sicht wird gegenüber der Betrachtung als Logik erster Stufe bevorzugt, weil dadurch
abgeleitete Prädikate zu lokalen Variablen des Programms werden, anstatt globaler Teil von Interpretationen (Datenbanken) zu sein

\end{frame}


\begin{frame}\frametitle{Zusammenfassung und Ausblick}

Datalog erlaubt die Darstellung rekursiver Anfragen in Logik\bigskip

Anfragebeantwortung in Datalog\\
= logisches Schließen in Prädikatenlogik\
= Auswertungsproblem in Logik zweiter Stufe\\
(\ExpTime-vollständig, aber polynomiell bzgl. Datenbankgröße)\bigskip

Ableitungen in Datalog können berechnet und dargestellt werden:
\begin{itemize}
\item mit dem $T_P$-Operator
\item durch Ableitungsbäume
\end{itemize}\bigskip

\anybox{yellow}{
Was erwartet uns als nächstes?
\begin{itemize}
\item Gödel
\item Konsultation, Probeklausur und Prüfung
\end{itemize}
}

\end{frame}


% \begin{frame}[t]\frametitle{Bildrechte}
% 
% Folie \ref{frame_herbrand}: Fotografie von Natasha Artin Brunswick, 1931, CC-By 3.0
% 
% \end{frame}


\end{document}
