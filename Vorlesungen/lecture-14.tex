\documentclass[aspectratio=1610,onlymath]{beamer}
% \documentclass[aspectratio=1610,onlymath,handout]{beamer}

\input{macros-lecture}

\defineTitle{14}{Modelltheorie und logisches Schließen}{14. Juni 2021}

\begin{document}

\maketitle

\begin{frame}\frametitle{Rückblick: Logelei}

Wir kehren zurück auf das Inselreich mit Menschen von
Typ W (Wahrheitssager) und Typ L (Lügner).
\bigskip

Smullyan\footnote{R. Smullyan: A Beginner's Guide to Mathematical Logic, Dover 2014} fragte die Bewohner nach ihren Rauchgewohnheiten.

\begin{itemize}
\item Auf \redalert{Insel A} antwortete jeder der Bewohner:\\ \alert{"`Jeder, der hier von Typ W ist, raucht."'}\\
%
\item Auf \redalert{Insel B} antwortete jeder der Bewohner:\\ \alert{"`Einige von uns hier sind von Typ L und rauchen."'}
%
\item Auf \redalert{Insel C} hatten alle den gleichen Typ und jeder sagte:\\ \alert{"`Falls ich rauche, dann raucht jeder hier."'}
%
\item Auf \redalert{Insel D} hatten alle den gleichen Typ und jeder sagte:\\ \alert{"`Einige hier rauchen, aber ich nicht."'}
\end{itemize}

Was können wir jeweils über die Bewohner und ihre Gewohnheiten schließen?

\end{frame}



\begin{frame}\frametitle{Prädikatenlogik: Syntax}

Wir betrachten unendliche, disjunkte Mengen von Variablen $\Slang{V}$, Konstanten $\Slang{C}$ und Prädikatensymbolen $\Slang{P}$.\medskip

\defbox{Ein prädikatenlogisches \redalert{Atom} ist ein Ausdruck $p(t_1,\ldots,t_n)$
für ein $n$-stelliges Prädikatensymbol $p\in\Slang{P}$ und Terme $t_1,\ldots,t_n\in\Slang{V}\cup\Slang{C}$.

Die Menge der \redalert{prädikatenlogische Formeln} ist induktiv definiert:
\begin{itemize}
\item Jedes Atom $p(t_1,\ldots,t_n)$ ist eine prädikatenlogische Formel
\item Wenn $x\in\Slang{V}$ eine Variable und $F$ und $G$ prädikatenlogische Formeln sind, dann sind auch die folgenden prädikatenlogische Formeln:
	\begin{itemize}
	\item $\neg F$: \redalert{Negation}, "`nicht $F$"'
	\item $(F\wedge G)$: \redalert{Konjunktion}, "`$F$ und $G$"'
	\item $(F\vee G)$: \redalert{Disjunktion}, "`$F$ oder $G$"'
	\item $(F\to G)$: \redalert{Implikation}, "`$F$ impliziert $G$"'
	\item $(F\leftrightarrow G)$: \redalert{Äquivalenz}, "`$F$ ist äquivalent zu $G$"'
	\item $\exists x. F$: \redalert{Existenzquantor}, "`für ein $x$ gilt $F$"'
	\item $\forall x. F$: \redalert{Allquantor}, "`für alle $x$ gilt $F$"'
	\end{itemize}
\end{itemize}
}

\end{frame}

\sectionSlide{Semantik der Prädikatenlogik}

\begin{frame}\frametitle{Interpretationen und Zuweisungen}

Die Wertzuweisung der Aussagenlogik wird also durch \alert{Interpretationen}
und \alert{Zuweisungen} für Variablen ersetzt.
\smallskip

\defbox{Eine \redalert{Interpretation} $\Inter$ ist ein Paar $\tuple{\Delta^\Inter,\cdot^\Inter}$ bestehend aus einer nichtleeren Grundmenge von Elemente $\Delta^\Inter$ (der \redalert{Domäne}) und einer \redalert{Interpretationsfunktion} $\cdot^\Inter$, welche:
\begin{itemize}
\item jede Konstante $a\in\Slang{C}$ auf ein Element $a^\Inter\in\Delta^\Inter$ und
\item jedes $n$-stellige Prädikatensymbol $p\in\Slang{P}$ auf eine Relation $p^\Inter\subseteq(\Delta^\Inter)^n$
\end{itemize}
abbildet.
}

\defbox{Eine \redalert{Zuweisung} $\Zuweisung$ für eine Interpretation $\Inter$ ist eine Funktion $\Zuweisung:\Slang{V}\to\Delta^\Inter$, die Variablen auf Elemente der Domäne abbildet. Für $x\in\Slang{V}$ und $\delta\in\Delta^\Inter$ schreiben wir \redalert{$\Zuweisung [x\mapsto \delta]$} für die Zuweisung, die $x$ auf $\delta$ und alle anderen Variablen $y\neq x$ auf $\Zuweisung(y)$ abbildet.}

\end{frame}

\begin{frame}\frametitle{Atome interpretieren}

Wir bestimmen dementsprechend die Wahrheit von Atomen unter einer Interpretation und
Zuweisung:

\defbox{Sei $\Inter$ eine Interpretation und $\Zuweisung$ eine Zuweisung für $\Inter$.
\begin{itemize}
\item Für eine Konstante $c$ definieren wir $c^{\Inter,\Zuweisung}=c^\Inter$
\item Für eine Variable $x$ definieren wir $x^{\Inter,\Zuweisung}=\Zuweisung(x)$
\end{itemize}
Für ein Atom $p(t_1,\ldots,t_n)$ setzen wir sodann:
\begin{itemize}
\item $p(t_1,\ldots,t_n)^{\Inter,\Zuweisung}=\mytrue$ wenn $\tuple{t_1^{\Inter,\Zuweisung},\ldots,t_n^{\Inter,\Zuweisung}}\in p^\Inter$ und
\item $p(t_1,\ldots,t_n)^{\Inter,\Zuweisung}=\myfalse$ wenn $\tuple{t_1^{\Inter,\Zuweisung},\ldots,t_n^{\Inter,\Zuweisung}}\notin p^\Inter$.
\end{itemize}
}\bigskip\pause

\anybox{strongyellow}{
\emph{Achtung!} Wir verwenden Interpretationen und Zuweisungen auf zwei Ebenen, 
die man nicht verwechseln sollte:
\begin{enumerate}[(1)]
\item um Terme $t$ auf Elemente $t^{\Inter,\Zuweisung}\in\Delta^\Inter$ abzubilden
\item um Atome $A$ auf Wahrheitswerte $A^{\Inter,\Zuweisung}\in\{\myfalse,\mytrue\}$ abzubilden
\end{enumerate}}

\end{frame}

\begin{frame}\frametitle{Formeln interpretieren}

\defbox{Eine Interpretation $\Inter$ und eine Zuweisung $\Zuweisung$ für
$\Inter$ \redalert{erfüllen} eine Formel $F$, in Symbolen \redalert{$\Inter,\Zuweisung\models F$}, wenn
eine der folgenden rekursiven Bedingungen gilt:%\smallskip

\narrowcentering{\footnotesize\begin{tabular}{rll}
\rowcolor{darkred!70!gray}
\textcolor{white}{Formel $F$} & \textcolor{white}{$\Inter,\Zuweisung\models F$ wenn:} & \textcolor{white}{$\Inter,\Zuweisung\not\models F$ wenn:}\\
$F$ Atom & $F^{\Inter,\Zuweisung}=\mytrue$ & $F^{\Inter,\Zuweisung}=\myfalse$\\
\rowcolor{lightred!20}
$\neg G$ & $\Inter,\Zuweisung\not\models G$ & $\Inter,\Zuweisung\models G$\\
$(G_1\wedge G_2)$ & $\Inter,\Zuweisung\models G_1$ und $\Inter,\Zuweisung\models G_2$ & $\Inter,\Zuweisung\not\models G_1$ oder $\Inter,\Zuweisung\not\models G_2$\\
\rowcolor{lightred!20}
$(G_1\vee G_2)$ & $\Inter,\Zuweisung\models G_1$ oder $\Inter,\Zuweisung\models G_2$ & $\Inter,\Zuweisung\not\models G_1$ und $\Inter,\Zuweisung\not\models G_2$\\
$(G_1\,{\to}\, G_2)$ & $\Inter,\Zuweisung\not\models G_1$ oder $\Inter,\Zuweisung\models G_2$ & $\Inter,\Zuweisung\models G_1$ und $\Inter,\Zuweisung\not\models G_2$\\
\rowcolor{lightred!20}
$(G_1\,{\leftrightarrow}\,G_2)$ & $\Inter,\Zuweisung\models G_1$ und $\Inter,\Zuweisung\models G_2$ & $\Inter,\Zuweisung\models G_1$ und $\Inter,\Zuweisung\not\models G_2$\\%[-1ex]
\rowcolor{lightred!20}
	& \multicolumn{1}{c}{\raisebox{1ex}{oder}} & \multicolumn{1}{c}{\raisebox{1ex}{oder}} \\[-2ex]
\rowcolor{lightred!20}
	& $\Inter,\Zuweisung\not\models G_1$ und $\Inter,\Zuweisung\not\models G_2$ & $\Inter,\Zuweisung\not\models G_1$ und $\Inter,\Zuweisung\models G_2$\\
%
$\forall x.G$ & $\Inter,\Zuweisung[x\mapsto\delta]\models G$ & $\Inter,\Zuweisung[x\mapsto\delta]\not\models G$\\[-0.5ex]
 & für alle $\delta\in\Delta^\Inter$ & für mindestens ein $\delta\in\Delta^\Inter$
\\
%
\rowcolor{lightred!20}
$\exists x.G$ & $\Inter,\Zuweisung[x\mapsto\delta]\models G$ & $\Inter,\Zuweisung[x\mapsto\delta]\not\models G$\\[-0.5ex]
\rowcolor{lightred!20}
 & für mindestens ein $\delta\in\Delta^\Inter$ & für alle $\delta\in\Delta^\Inter$
	\\[-1ex]
\\[-3.25ex]
\end{tabular}}

~
}

\end{frame}

\begin{frame}\frametitle{Beispiel}

\alert{"`Null ist eine natürliche Zahl und jede natürliche Zahl hat einen Nachfolger, der
ebenfalls eine natürliche Zahl ist."'}
\[ F=\textsf{NatNum}(\textsf{null})\wedge \forall x.\Big(\textsf{NatNum}(x)\to\exists y.\big(\textsf{succ}(x,y)\wedge \textsf{NatNum}(y)\big)\Big)\]

Wir betrachten eine Interpretation $\Inter$ mit
\begin{itemize}
\item $\Delta^\Inter=\mathbb{R}$ die Menge der reellen Zahlen
\item $\textsf{null}^\Inter=0$
\item $\textsf{NatNum}^\Inter=\mathbb{N}\subseteq\mathbb{R}$ die Menge der natürlichen Zahlen
\item $\textsf{succ}^\Inter=\{\tuple{d,e}\mid d,e\in\mathbb{R},d<e\}$
\end{itemize}\bigskip

Dann gilt $\Inter\models F$ (unter jeder beliebigen Zuweisung).

\anybox{strongyellow}{\emph{Notation:} Bei der Interpretation von Sätzen (Formeln ohne freie Variablen) spielen Zuweisungen keine Rolle. Wir schreiben sie in diesem Fall nicht.}

\end{frame}

\begin{frame}\frametitle{Logik auf Sätzen}

% Wie im vorigen Beispiel interessieren uns meist nur Sätze (geschlossene Formeln)
% \bigskip

\alert{Wie im vorigen Beispiel interessieren uns oft nur Sätze:}
\begin{itemize}
\item In den meisten Anwendungen arbeitet man nur mit Sätzen
\item Dann genügt es, Interpretationen zu betrachten
\item Zuweisungen sind in diesem Fall ein technisches Hilfsmittel zur Definition der Bedeutung von Sätzen
\end{itemize}

Eine Menge von Sätzen wird oft \redalert{Theorie} genannt.\pause

\examplebox{\emph{Beispiel:} Der Begriff stammt aus der Mathematik. Die Theorie der partiellen Ordnungen kann man z.B. wie folgt definieren:\vspace{-0.5ex}
\begin{align*}
\forall x. & (x\preceq x) & \text{Reflexivität}\\
\forall x,y,z. & \big( (x\preceq y \wedge y\preceq z) \to x\preceq z \big)& \text{Transitivität}\\
\forall x,y. & \big(( x\preceq y \wedge y\preceq x )\to x\approx y\big) & \text{Antisymmetrie}
\end{align*}\vspace{-3ex}

Dies definiert die Eigenschaften eines binären Prädikates $\preceq$ (hier infix geschrieben). Dabei verwenden wir zudem ein Gleichheitsprädikat $\approx$ (dazu später mehr).
}

\end{frame}


\sectionSlide{Semantische Grundbegriffe}


\begin{frame}\frametitle{Modelltheorie}

\emph{Wie definiert man logische Semantik "`modelltheoretisch"'?}\bigskip

\begin{tabular}{@{}p{0.7cm}p{3.5cm}p{2.0cm}p{2.3cm}@{}}
	& & \only<2->{\emph{Aussagenl.}} & \only<3->{\emph{Prädikatenl.}} \\
\rowcolor{lightred!30}
\cellcolor{darkred!70!gray}
\raisebox{-1cm}{\rotatebox{90}{\textcolor{white}{Formeln}}}
	& abzählbare Menge syntaktischer Ausdrücke\newline ~
	& \only<2->{Aussagen\-logische \mbox{Formeln}}
	& \only<3->{Prädikaten\-logische\newline Sätze}\\
\rowcolor{lightblue!30}
\cellcolor{darkblue!70!gray}
\raisebox{-1cm}{\rotatebox{90}{\textcolor{white}{Modelle}}}
	& Menge semantischer\newline \mbox{Interpretationen}\newline ~
	& \only<2->{Wert\-zuweisungen}
	& \only<3->{Prädikaten\-logische\newline Interpretationen}\\%[-3ex]
\rowcolor{lightgreen!30}
\cellcolor{darkgreen!70!gray}
\raisebox{-1.5cm}{\rotatebox{90}{\begin{minipage}{1.5cm}\textcolor{white}{Erfüllungs\-relation~$\models$}\end{minipage}}}
	& Beziehung zwischen\newline Modellen \& Formeln: In welchen Modellen sind welche Formeln wahr?
	& \only<2->{Aussagen\-logische Erfüllungs\-relation}
	& \only<3->{Prädikaten\-logische Erfüllungs\-relation}\\
\end{tabular}

\end{frame}

\begin{frame}\frametitle{Modelltheorie}

\emph{Wie definiert man logische Semantik "`modelltheoretisch"'?}\bigskip

\begin{tabular}{@{}p{0.7cm}p{3.5cm}p{2.0cm}p{2.3cm}p{3.5cm}@{}}
	& & {\emph{Aussagenl.}} & {\emph{Prädikatenl.}} & {\emph{Prädikatenl. (offen)}}\\
\rowcolor{lightred!30}
\cellcolor{darkred!70!gray}
\raisebox{-1cm}{\rotatebox{90}{\textcolor{white}{Formeln}}}
& abzählbare Menge syntaktischer Ausdrücke\newline ~
	& {Aussagen\-logische \mbox{Formeln}}
	& {Prädikaten\-logische\newline Sätze}
	& {Prädikaten\-logische\newline Formeln (offen oder\newline geschlossen)}\\
\rowcolor{lightblue!30}
\cellcolor{darkblue!70!gray}
\raisebox{-1cm}{\rotatebox{90}{\textcolor{white}{Modelle}}}
	& Menge semantischer\newline \mbox{Interpretationen}\newline ~
	& {Wert\-zuweisungen}
	& {Prädikaten\-logische\newline Interpretationen}
	& {Prädikatenlogische\newline Interpretationen\newline + Zuweisungen}\\%[-3ex]
\rowcolor{lightgreen!30}
\cellcolor{darkgreen!70!gray}
\raisebox{-1.5cm}{\rotatebox{90}{\begin{minipage}{1.5cm}\textcolor{white}{Erfüllungs\-relation~$\models$}\end{minipage}}}
	& Beziehung zwischen\newline Modellen \& Formeln: In welchen Modellen sind welche Formeln wahr?
	& {Aussagen\-logische Erfüllungs\-relation}
	& {Prädikaten\-logische Erfüllungs\-relation}
	& {Prädikaten\-logische Erfüllungs\-relation}\\
\end{tabular}

\end{frame}

\begin{frame}\frametitle{Modelltheorie: Intuition}

Die Modelltheorie einer Logik legt (ziemlich abstrakt) fest, worüber die Logik etwas aussagt:
\begin{itemize}
\item \alert{Formeln:} Behauptungen, die wahr oder falsch sein können
\item \alert{Modelle:} Mögliche Welten, in denen manche Behauptungen gelten und andere nicht
\end{itemize}\pause

\begin{center}
\begin{tikzpicture}[decoration=penciline, decorate]
% \draw[help lines] (0,0) grid (7,2);

\node (models) [rectangle,rounded corners=5ex, minimum width=2cm, minimum height=4cm,fill=lightblue!30,thick] at (2,-2) {};

\node (formulae) [rectangle,rounded corners=5ex, minimum width=2cm, minimum height=4cm,fill=lightred!30,thick] at (6,-2) {};

\node[rectangle,rounded corners=5ex] at (4,0.5) {\Large\textcolor{darkgreen}{$\models$}};
\node[rectangle,rounded corners=5ex] at (2,0.5) {\Large\textcolor{darkblue}{Modelle}};
\node[rectangle,rounded corners=5ex] at (6,0.5) {\Large\textcolor{darkred}{Formeln}};

\node (i1) [fill, draw, circle, minimum width=3pt, inner sep=0pt] at (2.5,-1) {};
\node[above left=-1pt of {i1}, outer sep=0pt] {\footnotesize$\Inter_1$};
\node (i2) [fill, draw, circle, minimum width=3pt, inner sep=0pt] at (1.8,-1.7) {};
\node[above left=-1pt of {i2}, outer sep=0pt] {\footnotesize$\Inter_2$};
\node (i3) [fill, draw, circle, minimum width=3pt, inner sep=0pt] at (2.3,-2.6) {};
\node[above left=-1pt of {i3}, outer sep=0pt] {\footnotesize$\Inter_3$};
\node (i4) [fill, draw, circle, minimum width=3pt, inner sep=0pt] at (2.1,-3.4) {};
\node[above left=-1pt of {i4}, outer sep=0pt] {\footnotesize$\Inter_4$};

\node (f1) [fill, draw, circle, minimum width=3pt, inner sep=0pt] at (5.7,-0.9) {};
\node[above right=-1pt of {f1}, outer sep=0pt] {\footnotesize$F_1$};
\node (f2) [fill, draw, circle, minimum width=3pt, inner sep=0pt] at (5.9,-1.6) {};
\node[above right=-1pt of {f2}, outer sep=0pt] {\footnotesize$F_2$};
\node (f3) [fill, draw, circle, minimum width=3pt, inner sep=0pt] at (6.3,-2.5) {};
\node[above right=-1pt of {f3}, outer sep=0pt] {\footnotesize$F_3$};
\node (f4) [fill, draw, circle, minimum width=3pt, inner sep=0pt] at (5.7,-3.3) {};
\node[above right=-1pt of {f4}, outer sep=0pt] {\footnotesize$F_4$};

\path[-,line width=0.5mm,darkgreen!70](i1) edge[decorate] (f1);
\path[-,line width=0.5mm,darkgreen!70](i2) edge[decorate] (f1);

\path[-,line width=0.5mm,darkgreen](i1) edge[decorate] (f2);
\path[-,line width=0.5mm,darkgreen](i2) edge[decorate] (f2);
\path[-,line width=0.5mm,darkgreen](i3) edge[decorate] (f2);
\path[-,line width=0.5mm,darkgreen](i4) edge[decorate] (f2);

\path[-,line width=0.5mm,darkgreen!50](i2) edge[decorate] (f3);
\path[-,line width=0.5mm,darkgreen!50](i3) edge[decorate] (f3);

% \path[->,line width=0.5mm](-1,0) edge (q0);
% \path[->,line width=0.5mm](q0) edge node[above] {$\blank\mapsto\Sterm{a},R$} (q1);
% \path[->,line width=0.5mm](q1) edge node[above] {$\blank\mapsto\Sterm{b},N$} (q2);
% \path[->,line width=0.5mm](q2) edge[bend left] node[below] {$\Sterm{b}\mapsto\Sterm{a},R$} (q0);
\end{tikzpicture}
\end{center}

\end{frame}

\begin{frame}\frametitle{Tautologien und Widersprüche}

Man unterscheidet Typen von Formeln nach ihren Modellen:

\begin{itemize}
\item \redalert{allgemeingültig (tautologisch):} Eine Formel, die in allen Modellen wahr ist
\item \redalert{widersprüchlich (inkonsistent, unerfüllbar):} Eine Formel, die in keinem Modell wahr \ghost{ist}
\item \redalert{erfüllbar (konsistent):} Eine Formel, die in einem Modell wahr ist
\item \redalert{widerlegbar:} Eine Formel, die in einem Modell falsch ist
\end{itemize}\pause

\begin{center}
\begin{tikzpicture}[decoration=penciline, decorate]
\pgfmathsetseed{7729}
% \draw[help lines] (0,0) grid (7,2);

\node (models) [rectangle,rounded corners=5ex, minimum width=2cm, minimum height=4cm,fill=lightblue!30,thick] at (2,-2) {};

\node (formulae) [rectangle,rounded corners=5ex, minimum width=2cm, minimum height=4cm,fill=lightred!30,thick] at (6,-2) {};
% 
% \node[rectangle,rounded corners=5ex] at (4,0.5) {\Large\textcolor{darkgreen}{$\models$}};
% \node[rectangle,rounded corners=5ex] at (2,0.5) {\Large\textcolor{darkblue}{Modelle}};
% \node[rectangle,rounded corners=5ex] at (6,0.5) {\Large\textcolor{darkred}{Formeln}};

\node (i1) [fill, draw, circle, minimum width=3pt, inner sep=0pt] at (2.5,-1) {};
\node[above left=-1pt of {i1}, outer sep=0pt] {\footnotesize$\Inter_1$};
\node (i2) [fill, draw, circle, minimum width=3pt, inner sep=0pt] at (1.8,-1.7) {};
\node[above left=-1pt of {i2}, outer sep=0pt] {\footnotesize$\Inter_2$};
\node (i3) [fill, draw, circle, minimum width=3pt, inner sep=0pt] at (2.3,-2.6) {};
\node[above left=-1pt of {i3}, outer sep=0pt] {\footnotesize$\Inter_3$};
\node (i4) [fill, draw, circle, minimum width=3pt, inner sep=0pt] at (2.1,-3.4) {};
\node[above left=-1pt of {i4}, outer sep=0pt] {\footnotesize$\Inter_4$};

\node (f1) [fill, draw, circle, minimum width=3pt, inner sep=0pt] at (5.7,-0.9) {};
\visible<-2>{\node[above right=-1pt of {f1}, outer sep=0pt] {\footnotesize$F_1$\phantom{g}};}
\visible<3->{\node[above right=-1pt of {f1}, outer sep=0pt] {\footnotesize$F_1$: erfüllbar, widerlegbar};}

\node (f2) [fill, draw, circle, minimum width=3pt, inner sep=0pt] at (5.9,-1.6) {};
\visible<-3>{\node[above right=-1pt of {f2}, outer sep=0pt] {\footnotesize$F_2$\phantom{g}};}
\visible<4->{\node[above right=-1pt of {f2}, outer sep=0pt] {\footnotesize$F_2$: erfüllbar, allgemeingültig};}

\node (f3) [fill, draw, circle, minimum width=3pt, inner sep=0pt] at (6.3,-2.5) {};
\visible<-4>{\node[above right=-1pt of {f3}, outer sep=0pt] {\footnotesize$F_3$ \phantom{g}};}
\visible<5->{\node[above right=-1pt of {f3}, outer sep=0pt] {\footnotesize$F_3$: erfüllbar, widerlegbar};}

\node (f4) [fill, draw, circle, minimum width=3pt, inner sep=0pt] at (5.7,-3.3) {};
\visible<-5>{\node[above right=-1pt of {f4}, outer sep=0pt] {\footnotesize$F_4$\phantom{g}};}
\visible<6->{\node[above right=-1pt of {f4}, outer sep=0pt] {\footnotesize$F_4$: unerfüllbar, widerlegbar};}

\path[-,line width=0.5mm,darkgreen!70](i1) edge[decorate] (f1);
\path[-,line width=0.5mm,darkgreen!70](i2) edge[decorate] (f1);

\path[-,line width=0.5mm,darkgreen](i1) edge[decorate] (f2);
\path[-,line width=0.5mm,darkgreen](i2) edge[decorate] (f2);
\path[-,line width=0.5mm,darkgreen](i3) edge[decorate] (f2);
\path[-,line width=0.5mm,darkgreen](i4) edge[decorate] (f2);

\path[-,line width=0.5mm,darkgreen!50](i2) edge[decorate] (f3);
\path[-,line width=0.5mm,darkgreen!50](i3) edge[decorate] (f3);

% \path[->,line width=0.5mm](-1,0) edge (q0);
% \path[->,line width=0.5mm](q0) edge node[above] {$\blank\mapsto\Sterm{a},R$} (q1);
% \path[->,line width=0.5mm](q1) edge node[above] {$\blank\mapsto\Sterm{b},N$} (q2);
% \path[->,line width=0.5mm](q2) edge[bend left] node[below] {$\Sterm{b}\mapsto\Sterm{a},R$} (q0);
\end{tikzpicture}
\end{center}

\end{frame}

\begin{frame}\frametitle{Logisches Schließen}

Aus der Modelltheorie ergibt sich, was \alert{logisches Schließen} genau bedeutet und
welche Schlüsse man ziehen darf:\bigskip

\defbox{\vspace{-1ex}
\begin{itemize}
\item Wenn $\Inter\models F$, dann nennt man $\Inter$ ein \redalert{Modell für} die Formel $F$ und man sagt $\Inter$ \redalert{erfüllt} $F$.
\item $\Inter$ ist ein \redalert{Modell für eine Formelmenge} $\mathcal{T}$, in Symbolen \redalert{$\Inter\models\mathcal{T}$}, wenn $\Inter\models F$ für jede Formel $F\in\mathcal{T}$.
\item Eine Formel $F$ ist eine \redalert{logische Konsequenz} aus einer Formel(menge) $G$, in Symbolen \redalert{$G\models F$}, wenn jedes Modell von $G$ auch ein Modell von $F$ ist, d.h. $\Inter\models G$ impliziert $\Inter\models F$.\\
\emph{Sonderfall:} Ist $F$ eine Tautologie, dann schreiben wir $\models F$
\item Zwei Formel(mengen) $F$ und $G$ sind \redalert{semantisch äquivalent}, in Symbolen
$F\equiv G$, wenn sie die gleichen Modelle haben, d.h. wenn $\Inter\models F$ gdw. $\Inter\models G$ für alle Modelle $\Inter$.
\end{itemize}
}

\end{frame}

\begin{frame}\frametitle{Beispiel: Logisches Schließen}

\begin{center}
\begin{tikzpicture}[decoration=penciline, decorate]
\pgfmathsetseed{7729}
% \draw[help lines] (0,0) grid (7,2);

\node (models) [rectangle,rounded corners=5ex, minimum width=2cm, minimum height=3.5cm,fill=lightblue!30,thick] at (2,-2) {};

\node (formulae) [rectangle,rounded corners=5ex, minimum width=2cm, minimum height=3.5cm,fill=lightred!30,thick] at (6,-2) {};

\visible<4->{\node [rectangle,rounded corners=2ex, minimum width=1.8cm, minimum height=1.8cm,fill=lightred!50,thick] at (6,-1.9) {};}
\visible<2->{\node [rectangle,rounded corners=1.5ex, minimum width=0.8cm, minimum height=0.7cm,fill=lightred!80,thick] at (6.4,-2.3) {};}
\visible<3->{\node [rectangle,rounded corners=2ex, minimum width=1.8cm, minimum height=1.7cm,fill=lightblue!60,thick] at (2,-2) {};}
% 
% \node[rectangle,rounded corners=5ex] at (4,0.5) {\Large\textcolor{darkgreen}{$\models$}};
% \node[rectangle,rounded corners=5ex] at (2,0.5) {\Large\textcolor{darkblue}{Modelle}};
% \node[rectangle,rounded corners=5ex] at (6,0.5) {\Large\textcolor{darkred}{Formeln}};

\node (i1) [fill, draw, circle, minimum width=3pt, inner sep=0pt] at (2.5,-1) {};
\node[above left=-1pt of {i1}, outer sep=0pt] {\footnotesize$\Inter_1$};
\node (i2) [fill, draw, circle, minimum width=3pt, inner sep=0pt] at (1.8,-1.7) {};
\node[above left=-1pt of {i2}, outer sep=0pt] {\footnotesize$\Inter_2$};
\node (i3) [fill, draw, circle, minimum width=3pt, inner sep=0pt] at (2.3,-2.6) {};
\node[above left=-1pt of {i3}, outer sep=0pt] {\footnotesize$\Inter_3$};
\node (i4) [fill, draw, circle, minimum width=3pt, inner sep=0pt] at (2.1,-3.4) {};
\node[above left=-1pt of {i4}, outer sep=0pt] {\footnotesize$\Inter_4$};

\node (f1) [fill, draw, circle, minimum width=3pt, inner sep=0pt] at (5.7,-0.9) {};
\node[above right=-1pt of {f1}, outer sep=0pt] {\footnotesize$F_1$};
\node (f2) [fill, draw, circle, minimum width=3pt, inner sep=0pt] at (5.9,-1.6) {};
\node[above right=-1pt of {f2}, outer sep=0pt] {\footnotesize$F_2$};
\node (f3) [fill, draw, circle, minimum width=3pt, inner sep=0pt] at (6.3,-2.5) {};
\node[above right=-1pt of {f3}, outer sep=0pt] {\footnotesize$F_3$};
\node (f4) [fill, draw, circle, minimum width=3pt, inner sep=0pt] at (5.7,-3.3) {};
\node[above right=-1pt of {f4}, outer sep=0pt] {\footnotesize$F_4$};

\path[-,line width=0.5mm,darkgreen!70](i1) edge[decorate] (f1);
\path[-,line width=0.5mm,darkgreen!70](i2) edge[decorate] (f1);

\path[-,line width=0.5mm,darkgreen](i1) edge[decorate] (f2);
\path[-,line width=0.5mm,darkgreen](i2) edge[decorate] (f2);
\path[-,line width=0.5mm,darkgreen](i3) edge[decorate] (f2);
\path[-,line width=0.5mm,darkgreen](i4) edge[decorate] (f2);

\path[-,line width=0.5mm,darkgreen!50](i2) edge[decorate] (f3);
\path[-,line width=0.5mm,darkgreen!50](i3) edge[decorate] (f3);

% \path[->,line width=0.5mm](-1,0) edge (q0);
% \path[->,line width=0.5mm](q0) edge node[above] {$\blank\mapsto\Sterm{a},R$} (q1);
% \path[->,line width=0.5mm](q1) edge node[above] {$\blank\mapsto\Sterm{b},N$} (q2);
% \path[->,line width=0.5mm](q2) edge[bend left] node[below] {$\Sterm{b}\mapsto\Sterm{a},R$} (q0);
\end{tikzpicture}
\end{center}

Was folgt aus $F_3$?\pause\pause
\begin{itemize}
\item Modelle von $F_3$ sind $\Inter_2$ und $\Inter_3$\pause
\item $\Inter_2$ und $\Inter_3$ sind Modelle von zwei Formeln: $F_3$ und $F_2$\pause
\end{itemize}
Anders gesagt: "`Immer wenn $F_3$ wahr ist, dann ist auch $F_2$ wahr."'\bigskip\pause

Es folgt also: $F_3\models F_2$


\end{frame}

\begin{frame}\frametitle{Eigenschaften der semantischen Äquivalenz}

Die aus der Aussagenlogik bekannten Eigenschaften von $\equiv$ gelten auch allgemein:\bigskip

\theobox{\emph{Satz:} $\equiv$ ist eine Äquivalenzrelation, d.h. reflexiv, symmetrisch und transitiv.}%
%
% \emph{Beweis:} $\equiv$ ist definiert als die Gleichheit der Modellmengen. Die gesuchten Eigenschaften ergeben sich, da auch die Relation $=$ auf Mengen eine Äquivalenzrelation ist.\qed
\medskip

% Weitere Eigenschaften folgen direkt aus den Definitionen:\medskip

\theobox{\emph{Satz:}
\begin{itemize}
\item Alle Tautologien sind semantisch äquivalent
\item Alle unerfüllbaren Formeln sind semantisch äquivalent
\end{itemize}
}\medskip

% \emph{Beweis:} Offensichtlich sind die Modellmengen jeweils gleich.\qed

\theobox{\emph{Satz:} Semantische Äquivalenz entspricht wechselseitiger logischer Konsequenz:\\[1ex]
% 
\narrowcentering{ $F\equiv G$ \hspace{5mm}genau dann, wenn\hspace{5mm} $F\models G$ und $G\models F$}
}

Die Behauptungen folgen jeweils direkt aus den Definitionen.

\end{frame}


\begin{frame}\frametitle{Inseln mit Lügnern und Wahrheitssagern}

Rückblick Logelei: "`Wir sind hier alle vom gleichen Typ."'
\begin{itemize}
\item "`Jeder Einwohner ist entweder Wahrheitssager oder Lügner."'\\
	$F_1=\alert{\forall x.\big((W(x)\wedge\neg L(x))\vee (L(x)\wedge\neg W(x))\big)}$
\item "`Auf dieser Insel haben alle den gleichen Typ."'\\
	$F_2=\alert{\forall x.W(x) \vee \forall x.L(x)}$
\end{itemize}\pause

Wir betrachten einige "`repräsentative"' Modelle von $F_1$, eine Formalisierung
der gegebenen Information und weitere Formeln:

\begin{center}
\begin{tikzpicture}[decoration=penciline, decorate]
% \draw[help lines] (0,0) grid (7,2);
\pgfmathsetseed{7729}

\node (models) [rectangle,rounded corners=5ex, minimum width=2cm, minimum height=3.6cm,fill=lightblue!30,thick] at (2,-1.8) {};

\node (formulae) [rectangle,rounded corners=5ex, minimum width=4cm, minimum height=3.6cm,fill=lightred!30,thick] at (7,-1.8) {};

\node (i1) [rectangle,rounded corners=2ex, minimum width=1.2cm, minimum height=0.7cm,fill=strongyellow,thick] at (2,-0.9) {};
\node (p11) [fill, draw, circle, minimum width=3pt, inner sep=0pt] at (1.85,-1) {};
\node[above left=-2pt of {p11}, outer sep=0pt] {\footnotesize$L$};
\node (p12) [fill, draw, circle, minimum width=3pt, inner sep=0pt] at (2.3,-1.05) {};
\node[above left=-2pt of {p12}, outer sep=0pt] {\footnotesize$L$};

\node (i2) [rectangle,rounded corners=2ex, minimum width=1.2cm, minimum height=0.7cm,fill=strongyellow,thick] at (2,-1.9) {};
\node (p21) [fill, draw, circle, minimum width=3pt, inner sep=0pt] at (1.85,-2) {};
\node[above left=-4pt of {p21}, outer sep=0pt] {\footnotesize$W$};
\node (p22) [fill, draw, circle, minimum width=3pt, inner sep=0pt] at (2.3,-2.05) {};
\node[above left=-2pt of {p22}, outer sep=0pt] {\footnotesize$L$};

\node (i3) [rectangle,rounded corners=2ex, minimum width=1.2cm, minimum height=0.7cm,fill=strongyellow,thick] at (2,-2.9) {};
\node (p31) [fill, draw, circle, minimum width=3pt, inner sep=0pt] at (1.85,-3) {};
\node[above left=-4pt of {p31}, outer sep=0pt] {\footnotesize$W$};
\node (p32) [fill, draw, circle, minimum width=3pt, inner sep=0pt] at (2.3,-3.05) {};
\node[above left=-4pt of {p32}, outer sep=0pt] {\footnotesize$W$};

\node (t) [rectangle,rounded corners=2ex, minimum width=3.2cm, minimum height=1.9cm,fill=lightred!60,thick] at (7,-1.2) {};
\node[right=-1pt of {t}, outer sep=0pt, align=left] {Gegebene\\Theorie};
\node (f1) [fill, draw, circle, minimum width=3pt, inner sep=0pt] at (5.7,-0.7) {};
\node[above right=-3pt of {f1}, outer sep=0pt] {\footnotesize$F_1$};
\node (f2) [fill, draw, circle, minimum width=3pt, inner sep=0pt] at (5.9,-1.2) {};
\node[above right=-1pt of {f2}, outer sep=0pt] {\footnotesize$\exists x.W(x)\to F_2$};
\node (f3) [fill, draw, circle, minimum width=3pt, inner sep=0pt] at (5.8,-1.9) {};
\node[above right=-1pt of {f3}, outer sep=0pt] {\footnotesize$\exists x.L(x)\to \neg F_2$};

\node (f4) [fill, draw, circle, minimum width=3pt, inner sep=0pt] at (5.7,-2.8) {};
\node[above right=-1pt of {f4}, outer sep=0pt] {\footnotesize$\forall x.L(x)$};
\node (f5) [fill, draw, circle, minimum width=3pt, inner sep=0pt] at (5.9,-3.4) {};
\node[above right=-1pt of {f5}, outer sep=0pt] {\footnotesize$\forall x.W(x)$};

\visible<3->{
\path[-,line width=0.5mm,darkgreen!60](i1) edge[decorate] (f1);
\path[-,line width=0.5mm,darkgreen!60](i2) edge[decorate] (f1);
\path[-,line width=0.5mm,darkgreen!60](i3) edge[decorate] (f1);
}

\visible<4->{
\path[-,line width=0.5mm,darkgreen!80](i1) edge[decorate] (f2);
\path[-,line width=0.5mm,darkgreen!80](i3) edge[decorate] (f2);
}

\visible<5->{
\path[-,line width=0.5mm,darkgreen](i2) edge[decorate] (f3);
\path[-,line width=0.5mm,darkgreen](i3) edge[decorate] (f3);
}

\visible<6->{
\path[-,line width=0.5mm,darkgreen](i1) edge[decorate] (f4);
}
\visible<7->{
\path[-,line width=0.5mm,darkgreen](i3) edge[decorate] (f5);
}
% \path[-,line width=0.5mm,darkgreen!70](i2) edge[decorate] (f1);
% 
% \path[-,line width=0.5mm,darkgreen](i1) edge[decorate] (f2);
% \path[-,line width=0.5mm,darkgreen](i2) edge[decorate] (f2);
% \path[-,line width=0.5mm,darkgreen](i3) edge[decorate] (f2);
% \path[-,line width=0.5mm,darkgreen](i4) edge[decorate] (f2);
% 
% \path[-,line width=0.5mm,darkgreen!50](i2) edge[decorate] (f3);
% \path[-,line width=0.5mm,darkgreen!50](i3) edge[decorate] (f3);

% \path[->,line width=0.5mm](-1,0) edge (q0);
% \path[->,line width=0.5mm](q0) edge node[above] {$\blank\mapsto\Sterm{a},R$} (q1);
% \path[->,line width=0.5mm](q1) edge node[above] {$\blank\mapsto\Sterm{b},N$} (q2);
% \path[->,line width=0.5mm](q2) edge[bend left] node[below] {$\Sterm{b}\mapsto\Sterm{a},R$} (q0);
\end{tikzpicture}
\end{center}
\vspace{-1.7ex}

\visible<8->{Aus der Theorie folgt $\forall x.W(x)$: "`Alle Inselbewohner sind von \ghost{Typ W."'}}

\end{frame}

\begin{frame}\frametitle{Das Problem des logischen Schließens}

Zwei praktisch wichtige Fragen:
\begin{enumerate}[(1)]
\item \redalert{Model Checking:}\\ Für ein gegebenes Modell $\Inter$ und eine Formel $F$, gilt $\Inter\models F$?
\item \redalert{Logische Folgerung (Entailment):}\\ Für gegebene Formel(menge)n $F$ und $G$, gilt $F\models G$?
\end{enumerate}\pause\bigskip

\alert{In der Aussagenlogik} ist beides relativ einfach lösbar:
\begin{enumerate}[(1)]
\item Berechne den Wahrheitswert unter einer Belegung (zeitlinear)
\item Betrachte alle möglichen Belegungen (exponentiell; \complclass{NP}-vollständig)
\end{enumerate}\pause\bigskip

\alert{In der Prädikatenlogik} ist das nicht so einfach:\\ siehe kommende Vorlesungen

\end{frame}

\begin{frame}\frametitle{Monotonie und Tautologie}

Aus der Definition von $\models$ folgt \redalert{Monotonie}:
\begin{itemize}
\item Mehr Sätze $\Leftrightarrow$ weniger Modelle
\item Je \alert{mehr Sätze} in einer logischen Theorie gegeben sind,\\
	desto \alert{weniger Modelle} erfüllen die gesamte Theorie,\\
	desto \alert{mehr Schlussfolgerungen} kann man aus ihr ziehen
\end{itemize}
Das heißt: \redalert{"`Mehr Annahmen führen zu mehr Schlussfolgerungen"'}
\bigskip\pause

Die Extremfälle dieses Prinzips sind:
\begin{itemize}
\item \alert{Tautologien:} sind in jedem Modell wahr und daher logische Konsequenz jeder Theorie
\item \alert{Unerfüllbare Formeln:} sind in keinem Modell wahr und haben daher alle anderen Sätze als Konsequenz
\end{itemize}

\end{frame}

\begin{frame}\frametitle{Modelltheorie ist allgemein gültig}

Was wir bisher über Modelltheorie gesagt haben, gilt für jede
Logik, deren Semantik auf einer Beziehung $\models$ von Modellen
zu einzelnen Formeln basiert:
\begin{itemize}
\item Aussagenlogik
\item Prädikatenlogik (nur Sätze und Interpretationen)
\item Prädikatenlogik (beliebige Formeln und Interpretationen+Zuweisungen)
\item Logik zweiter Stufe
\item Modal-, Temporal- und Beschreibungslogiken
\item Mehrwertige Logiken
\item Nichtklassische Logiken
\item \ldots
\end{itemize}

Andere Eigenschaften der Prädikatenlogik sind nicht ganz so allgemein.

\end{frame}

\sectionSlide{Prädikatenlogik und Aussagenlogik}


\begin{frame}\frametitle{Verhältnis zur Aussagenlogik}

Die Semantik der Operatoren $\neg$, $\wedge$, $\vee$, $\to$ und $\leftrightarrow$ ist
in Prädikatenlogik und Aussagenlogik gleich definiert:
\begin{itemize}
\item Wir ersetzen Wertzuweisungen $w$ durch\\
	Interpretationen $\Inter$ mit Zuweisungen $\Zuweisung$
\item Ansonsten ist die Definition der Semantik genau gleich
\end{itemize}
$\leadsto$ alle aussagenlogischen Gesetze gelten analog
\bigskip\pause

\examplebox{\emph{Beispiel:} Die De Morganschen Regeln gelten auch in der Prädikatenlogik, z.B.
$\Inter,\Zuweisung\models \neg(F\wedge G)$ genau dann, wenn
$\Inter,\Zuweisung\models (\neg F\vee \neg G)$, das heißt
$\neg(F\wedge G)\equiv (\neg F\vee \neg G)$.
}

Allgemein gelten alle bekannten aussagenlogischen Äquivalenzen

\end{frame}

\begin{frame}\frametitle{"`${\models} = {\to}$"' und "`${\equiv}={\leftrightarrow}$"'}

Auch die folgenden Sätze gelten analog zur Aussagenlogik\\ (siehe Formale Systeme, Vorlesung 22):\bigskip

\theobox{\emph{Satz (Deduktionstheorem):} Für jede Formelmenge $\mathcal{F}$ und Formeln $G$ und $H$ gilt\\
$\mathcal{F}\models G\to H$ genau dann, wenn $\mathcal{F}\cup\{G\}\models H$.}\bigskip

\theobox{\emph{Korollar:} $F\wedge G\models H$ genau dann, wenn $F\models G\to H$.}\bigskip

\theobox{\emph{Korollar:} $F\equiv G$ genau dann, wenn $\models F\leftrightarrow G$.}\bigskip

\redalert{Dennoch sind $\models$ und $\equiv$ nicht dasselbe wie $\to$ und $\leftrightarrow$:}
\begin{itemize}
\item $\models$ und $\equiv$ können sich auch auf (möglicherweise unendliche) Mengen von Formeln beziehen
\item $\to$ und $\leftrightarrow$ sind syntaktische Operatoren und können (eventuell geschachtelt) in Formeln auftreten
\end{itemize}

\end{frame}

\begin{frame}\frametitle{Das Ersetzungstheorem}

Auch das folgende intuitiv einleuchtende Ergebnis kann von der Aussagenlogik auf
die Prädikatenlogik übertragen werden:\bigskip

\theobox{\emph{Satz (Ersetzungstheorem):} Sei $F$ eine Formel mit einer Teilformel $G$. 
Wenn $G\equiv G'$ und wenn $F'$ aus $F$ gebildet werden kann, indem man ein beliebiges Vorkommen von $G$ in $F$ durch $G'$ ersetzt, dann gilt auch $F\equiv F'$.}\bigskip

Der detaillierte Beweis muss allerdings alle möglichen Formen von
Formeln betrachten (Induktion über Formelstruktur). Im Vergleich zur
Aussagenlogik müsste man also noch zeigen, dass die Ersetzung von
äquivalenten Formeln in $\exists x.G$ und $\forall x.G$ zulässig ist.

\end{frame}

\begin{frame}\frametitle{Rückblick: Aussagenlogische Äquivalenzen (1)}

\begin{align*}
\begin{split}
F\wedge G & \equiv G\wedge F\\
F\vee G & \equiv G\vee F
\end{split}
		& \text{\textcolor{devilscss}{Kommutativität}} \\[1ex]
%
\begin{split}
(F\wedge G)\wedge H & \equiv F\wedge (G\wedge H)\\
(F\vee G)\vee H & \equiv F\vee (G\vee H)
\end{split}
		& \text{\textcolor{devilscss}{Assoziativität}} \\[1ex]
%
\begin{split}
F\wedge (G\vee H) & \equiv (F\wedge G) \vee (F\wedge H)\\
F\vee (G\wedge H) & \equiv (F\vee G)\wedge (F\vee H)
\end{split}
		& \text{\textcolor{devilscss}{Distributivität}} \\[1ex]
%
\begin{split}
F\wedge F & \equiv F\\
F\vee F & \equiv F
\end{split}
		& \text{\textcolor{devilscss}{Idempotenz}} \\[1ex]
%
\begin{split}
F\wedge (F\vee G) & \equiv F\\
F\vee (F\wedge G) & \equiv F
\end{split}
		& \text{\textcolor{devilscss}{Absorption}}
\end{align*}

\end{frame}

\begin{frame}\frametitle{Rückblick: Aussagenlogische Äquivalenzen (2)}

\begin{align*}
\neg\neg F &\equiv F
		& \text{\textcolor{devilscss}{doppelte Negation}}\\[1ex]
%
\begin{split}
\neg(F\wedge G) & \equiv (\neg F\vee \neg G)\\
\neg(F\vee G) & \equiv (\neg F\wedge \neg G)
\end{split}
		& \text{\textcolor{devilscss}{De Morgansche Gesetze}} \\[1ex]
%
\begin{split}
F\wedge \top & \equiv F\\
F\vee \top & \equiv \top
\end{split}
		& \text{\textcolor{devilscss}{Gesetze mit $\top$}}\\[1ex]
%
\begin{split}
F\wedge \bot & \equiv \bot\\
F\vee \bot & \equiv F
\end{split}
		& \text{\textcolor{devilscss}{Gesetze mit $\bot$}}\\[1ex]
%
\begin{split}
\neg\top & \equiv \bot\\
\neg\bot & \equiv \top
\end{split}
		& 
\end{align*}

Dabei stellen wir wie zuvor $\top$ durch eine beliebige Tautologie (z.B. $p\vee\neg p$) und $\bot$ durch einen beliebigen Widerspruch (z.B. $p\wedge\neg p$) dar.

\end{frame}


\begin{frame}\frametitle{Aussagenlogik in Prädikatenlogik darstellen}

Aussagenlogische Atome $p$ kann man durch prädikatenlogische Atome $p()$
auffassen, wobei $p$ ein nullstelliges Prädikatensymbol ist.%
\bigskip

Sei $p\in\Slang{P}$ ein nullstelliges Prädikat\\[1ex]
\redalert{Welche Interpretationen $p^\Inter$ sind möglich?}\pause
\begin{itemize}
\item Laut Definition gilt $p^\Inter\subseteq(\Delta^\Inter)^0$\pause
\item $(\Delta^\Inter)^0$ enthält alle "`nullstelligen Tupel"'\\
$\leadsto$ es gibt aber nur ein einziges nullstelliges Tupel $\tuple{}$\pause
\item Also ist $p^\Inter\subseteq\{\tuple{}\}$:
	\begin{itemize}
	\item $p^\Inter=\{\tuple{}\}$ bedeutet $\Inter\models p()$ (\alert{"`Aussage wahr"'})
	\item $p^\Inter=\{\}$ bedeutet $\Inter\not\models p()$ (\alert{"`Aussage falsch"'})
	\end{itemize}
\end{itemize}

Deshalb kann man nullstellige Prädikate wie aussagenlogische Atome verwenden
\bigskip

In diesem Sinne ist die Aussagenlogik ein Teil der Prädikatenlogik

\end{frame}


\begin{frame}\frametitle{Auflösung: Logelei}

Im Inselreich der Menschen von Typ W und Typ L fragte 
Smullyan\footnote{R. Smullyan: A Beginner's Guide to Mathematical Logic, Dover 2014} die Bewohner nach ihren Rauchgewohnheiten:

\begin{itemize}
\item Auf \redalert{Insel A} antwortete jeder der Bewohner:\\ \alert{"`Jeder, der hier von Typ W ist, raucht."'}\\
\visible<2->{\redalert{Die Aussage stimmt und alle sind vom Typ W.}}
%
\item Auf \redalert{Insel B} antwortete jeder der Bewohner:\\ \alert{"`Einige von uns hier sind von Typ L und rauchen."'}\\
\visible<3->{\redalert{Alle sind vom Typ L und keiner raucht.}}
%
\item Auf \redalert{Insel C} hatten alle den gleichen Typ und jeder sagte:\\ \alert{"`Falls ich rauche, dann raucht jeder hier."'}\\
\visible<4->{\redalert{Alle sagen die Wahrheit; es rauchen alle oder keiner.}}
%
\item Auf \redalert{Insel D} hatten alle den gleichen Typ und jeder sagte:\\ \alert{"`Einige hier rauchen, aber ich nicht."'}\\
\visible<5->{\redalert{Alle lügen; es rauchen alle oder keiner.}}
%
\end{itemize}

\end{frame}

\begin{frame}\frametitle{Zusammenfassung und Ausblick}

Modelltheorie definiert logische Semantik aus der Beziehung von Formeln (Behauptungen) und Modellen (möglichen Welten)
\bigskip

% Die Modelltheorie der Prädikatenlogik basiert auf Interpretationen und Zuweisungen, aber bei Sätzen reichen Interpretationen
% \bigskip

Logisches Schließen ist die Berechnung (Überprüfung) einzelner Beziehungen der Form $\Inter\models F$ (Model Checking) bzw. $F\models G$ (Schlussfolgerung)\bigskip

Prädikatenlogik verallgemeinert Aussagenlogik und viele der dort gültigen Gesetze
\bigskip

\anybox{yellow}{
Was erwartet uns als Nächstes?
\begin{itemize}
\item Logisches Schließen: (Un)Entscheidbarkeit und Komplexität
\item Resolution für Prädikatenlogik
% \item Zweites Repetitorium
\end{itemize}
}

\end{frame}



% \begin{frame}[t]\frametitle{Literatur und Bildrechte}
% 
% \alert{Literatur}\bigskip
% 
% \begin{itemize}
% \item Richard J. Lorentz:
% \emph{Creating Difficult Instances of the Post Correspondence Problem.}
% Computers and Games 2000: 214--228
% \item John J. O'Connor, Edmund F. Robertson: \emph{Emil Leon Post.} MacTutor History of Mathematics archive, University of St Andrews. \url{http://www-history.mcs.st-andrews.ac.uk/Biographies/Post.html}
% \end{itemize}
% 
% \bigskip\bigskip
% 
% \alert{Bildrechte}\bigskip
% 
% Folie \ref{frame_post}: gemeinfrei
% 
% \end{frame}


\end{document}
